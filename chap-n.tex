\chapter{Revisiting a Randomness Beacon Protocol}
\label{cha:n}

Randomness Beacon \cite{RABIN1983256} is required in applications like e-voting \cite{10.5555/1496711.1496734} 
and anonymous messaging (\cite{180263},\cite{10.1145/2815400.2815417}) to provide fresh random values to all the 
parties. In 2020, Cascudo and David published ALBATROSS \cite{cryptoeprint:2020/644}, the state-of-the art randomness 
beacon protocol based on a PVSS as a building block where each party in the randomness beacon protocol acts as a dealer once, so that all 
parties can influence the output randomness. Interestingly, we observed that each party is expected to reveal 
their secrets (they secret shared as a dealer) as part of the randomness beacon protocol, but to prove that the 
secrets are valid and not just some random evaluations of the secret polynomial they have to reveal the whole 
secret polynomial itself. As a consequence, if some entity wants to verify the secrets' validity then they have to 
simulate the whole sharing phase of the underlying PVSS protocol, which is very expensive because all the rest of the 
parties are expected to do the simulation of that party as a dealer. For reference, if there are $n$ parties, then 
$n-1$ parties should simulate the sharing phase of the protocol, which in total is $\mathcal{O}(n^2)$ simulations.\par

In this chapter, we present our randomness beacon protocol in figures \ref{fig:randomness_beacon} and \ref{fig:randomness_beacon_cont} 
which in many cases is efficient than ALBATROSS. To put simply, we replaced the building block being PVSS with 
our PPPVSS $\Lambda_{RO}$ \ref{fig:packed-shamir-PPPVSS-ro}. In the subsequent sections, we will discuss the 
computational and communication costs of our protocol and compare it with the ALBATROSS. We will show that out protocol 
performs more efficient compared to ALBATROSS in many cases and also address the cases where we are not computationally efficient.
More interestingly, we will show that in terms of communication, we outperform ALBATROSS.

\begin{figure}[ht]
    \centering
    \begin{tcolorbox}[title=\textbf{Randomness Beacon using PPPVSS}, width=0.9\textwidth, colframe=blue!75!black, colback=blue!10, sharp corners]
        Our protocol with PPPVSS is run between a set $\mathcal{P}$ of $n$ 
        parties $P_1, \dots, P_n$ who have access to a public ledger where they 
        can post information for later verification. It is assumed that the 
        Setup phase of $\Pi_{PPPVSS}$ is already done and the public keys 
        $\text{pk}_i$ of each party $P_i$ along with $\{\mathbb{P}_i\}_{i=1}^{l}$ 
        being Commitment keys (or public keys of target people) to encrypt the 
        $l$ secrets are already registered in the ledger. In addition, the 
        parties have agreed on a Vandermonde $(n - 2t) \times (n - t)$-matrix 
        $M = M(\omega, n - 2t, n - t)$ with $\omega \in \mathbb{Z}_q^*$.

    \begin{enumerate}
        \item [1.]\textbf{Commit:} For $1 \leq j \leq n$:
        \begin{itemize}
            \item Shareholder $P_j$ executes the Distribution phase of the 
            PPPVSS as Dealer for $\ell = n - 2t$ secrets, publishing commitments 
            (/encryptions) of secrets, $y_{-(l-1)}^j, \dots, y_{-1}^j, y_0^j$, 
            and encryptions of shares $\{y_i^j\}_{i=1}^n$ along with 
            $\pi_{proof}^{j}$, which is a NIZK PoK for proving the correctness of 
            committed(/encrypted) secrets and encrypted secret shares on the 
            public ledger, also learning the secrets $h^{s_0^j},...,h^{s_{-(l-1)}^j}$ 
            and their corresponding exponents\\ $s_0^j, \dots, s_{-(l-1)}^j$.
        \end{itemize}
        
        \item [2.]\textbf{Reveal:}
        \begin{itemize}
            \item Each shareholder checks the validity of the proof 
            $\pi_{proof}^j$, i.e., the \textbf{verification phase of PPPVSS protocol}.
            \item After a set $\mathcal{C}$ containing at least $n-t$ 
            shareholders publish their shares in the public ledger, 
            $P_j\in\mathcal{C}$ reveals $l$ secrets.
            \item Every shareholder verifies the validity of secrets by 
            reproducing the commitments using the commitment keys (/public keys 
            of target people).
            \item At this point, if every party in $\mathcal{C}$ has opened 
            their secrets correctly, go to step 4' in Figure \ref{fig:9}. 
            Otherwise, proceed to step 3 in Figure \ref{fig:9}.
        \end{itemize}
    \end{enumerate}
    \end{tcolorbox}
    \caption{Commit and Reveal phase of the Randomness Beacon using PPPVSS}
    \label{fig:randomness_beacon}
\end{figure}
\begin{figure}[t!]
    \centering
    \begin{tcolorbox}[title=\textbf{Randomness Beacon using 3PVSS, $\Lambda_{RO}^{packed}$ (cont.)}, width=0.9\textwidth, colframe=blue!75!black, colback=blue!10, sharp corners]
        \begin{enumerate}
            \item [3.]\textbf{Recovery:} Let $\mathcal{C}_a$ be the set containing at most $t$ malicious shareholders(as Dealers) who did not open the exponents corresponding to their $\ell$ secrets, $\{h^{s_i^k}\}_{i=0}^{-(\ell-1)}$ for each $P_k\in\mathcal{C}_a$, in \textit{Reveal} phase.
            \begin{itemize}
                \item Every shareholder $P_j$ should decrypt the secret share of each malicious shareholder(Dealer) in $\mathcal{C}_a$, and give a DLEQ proof \ref{subsec:chaum-pedersen} which asserts that the decryption is performed correctly,i.e., each shareholder should perform the \textit{pessimistic} reconstruction phase of the 3PVSS $\Lambda_{RO}^{packed}$ for every shareholder(Dealer) who has not revealed the exponents corresponding to their secrets.
            \end{itemize}
            
            \item [4]\textbf{Output:}  Let $T$ be the $(n - t) \times \ell$ matrix with rows indexed by the shareholders in $\mathcal{C}$ and where the row corresponding to $P_a \in \mathcal{C}$ is $(h^{s_0^a} , . . . , h^{s_{-(\ell-1)}^a})$.
            \begin{itemize}
                \item Each computes the $\ell \times \ell$-matrix $R = M \circ T$ by applying FFTE to each column $T^{(j)}$ of $T$, resulting in column $R^{(j)}$ of $R$ (since $R^{(j)} = M \circ T^{(j)}$ and $M$ is Vandermonde) for $j \in [0, \ell - 1]$.
                \item Shareholders output the $\ell^2$ elements of $R$ as final randomness.
            \end{itemize}
        
            \item [4']\textbf{Alternative Output:}  if every party in $\mathcal{C}$ has opened her secrets correctly in step \textit{Reveal}, then:
            \begin{itemize}
                \item Shareholders compute $R = M \circ T$ in the following way:\\
                    Let $S$ be the $(n - t) \times \ell$ matrix with rows indexed by the shareholders in $\mathcal{C}$ and where the row
                    corresponding to $P_a \in\mathcal{C}$ is $(s_0^a,...,s_{-(\ell-1)}^a )$. Then each party computes $U = M \circ S \in\mathbb{Z}_q^{\ell\times \ell}$ (using the standard FFT in $\mathbb{Z}_q$ to compute each column) and $R = h^U$ .
                \item Shareholders output the $\ell^2$ elements of $R$ as final randomness.
            \end{itemize}
        \end{enumerate}
    \end{tcolorbox}
    \caption{Recovery and Output phase of the Randomness Beacon using 3PVSS}
    \label{fig:randomness_beacon_cont}
\end{figure}
\section{Computational Complexity}
\begin{table}[H]
\centering
\begin{tabular}{|p{3cm}|p{1.2cm}|p{2.5cm}|p{5.5cm}|p{2.5cm}|}
\hline
\textbf{Protocol}    & \textbf{Output size}    & 
\textbf{Commit}\textit{(by Dealer)} & \textbf{Reveal}\textit{(by 
shareholder)} & \textbf{Recovery} \textit{(by shareholder)}                                                           
\\ \hline
\textbf{ALBATROSS}, \textit{Honest case}    & $l^2$ & 
$(2n+l)[\mathbb{E}_x+\mathbb{P}_e]$ & \textbf{Share Verification - }  &  
\\
& & & $(n-1)n[2\mathbb{E}_x+\mathbb{P}_e]$ & \\
& & & \textbf{Secret Verification - } & \\ 
& & & $(n-1)(n+l)[\mathbb{E}_x+\mathbb{P}_e]$&- \\ \hline
\textbf{with PPPVSS}, \textit{Honest case}    & $l^2$  & 
$2(n+l)[\mathbb{E}_x+\mathbb{P}_e]$ & \textbf{Share Verification - } &  \\ 
& & & $(n-1)(n+l)[2\mathbb{E}_x+\mathbb{P}_e]$ &  \\ 
& & & &  \\
& & & \textbf{Secret Verification - } & \\ 
& & & $(n-1)l\mathbb{E}_x$ & -  \\ \hline
\textbf{ALBATROSS}, \textit{Robust case}    & $l^2$ & 
$(2n+l)[\mathbb{E}_x+\mathbb{P}_e]$ & \textbf{Share Verification - }  &  
\\
& & & $(n-1)n[2\mathbb{E}_x+\mathbb{P}_e]$ & \\
& & & \textbf{Secret Verification - } & \\ 
& & & $(n-t-1)(n+l)[\mathbb{E}_x+\mathbb{P}_e]$& 
$[3+4(n-t)]t\mathbb{E}_{x}$\\ \hline
\textbf{with PPPVSS}, \textit{Robust case}    & $l^2$  & 
$2(n+l)[\mathbb{E}_x+\mathbb{P}_e]$ & \textbf{Share Verification - } &  \\ 
& & & $(n-1)(n+l)[2\mathbb{E}_x+\mathbb{P}_e]$ &  \\
& & & \textbf{Secret Verification - } & \\ 
& & & $(n-t-1)l\mathbb{E}_x$ &   \\
& & & & $[3+4(n-t)]t\mathbb{E}_{x}$  \\ \hline

\end{tabular}
\caption{Computational cost of dealer and shareholders, 
$\mathbb{E}_x=$group exponentiation and $\mathbb{P}_e=$polynomial 
evaluation in group $G$ with order $q$, where $q$ is a large prime}
\label{tab:comp_alba_pppvss_no group mul}
\end{table}


See table \ref{tab:comp_alba_pppvss_no group mul} for an overview.
\begin{itemize}
    \item In ALBATROSS, a dealer(as a part of \textbf{commit}) should compute $n(\mathbb{E}_x+\mathbb{P}_e)$ commitments and to give a proof he should do an additional $n(\mathbb{P}_{e}+\mathbb{E}_{x})$. Also, on dealer should do $l(\mathbb{P}_e+\mathbb{E}_x)$ for computing secrets and keeping it to himself. In total dealer needs to do $(2n+l)[\mathbb{E}_x+\mathbb{P}_{e}]$.
        \begin{itemize}
            \item In \textbf{Reveal}, a verifier should compute $2n\mathbb{E}_{x}$ which internally requires additional $n\mathbb{P}_{e}$, i.e., in total it requires $(n-1)n(2\mathbb{E}_{x}+\mathbb{P}_{e})$ computations for each verifier.
            \begin{itemize}
                \item In \textbf{Robust case} where $t$ dealers do not open their polynomials, a verifier should verify $n-t$ polynomials of honest dealers, i.e., for each honest dealer, a verifier has to do $n\mathbb{P}_e$ to evaluate secret share exponents and does $n\mathbb{E}_x$ to get secret shares and cross checks them in the public ledger. Also, finally the verifier computes $l\mathbb{P}_e$ to get secret exponents and get $l$ secrets by doing $l\mathbb{E}_x$. As there are $n-t$ honest dealers, the verifier has to compute $(n-t)(n+l)(\mathbb{E}_x+\mathbb{P}_e)$.
                \item  In \textbf{Honest case}, everyone would have been honest and so each verifier has to do $(n-1)(n+l)(\mathbb{E}_x+\mathbb{P}_e)$.
            \end{itemize}
            \item \textbf{Recovery} phase only exists if some party does not 
            open the polynomial leading to PVSS reconstruction phase, in the 
            worst case there should be reconstruction for the secrets of $t$ 
            malicious parties. Given a malicious shareholder who has not opened 
            the secret polynomial, each shareholder/re-constructor has to 
            decrypt their share, which requires $1\mathbb{E}_{x}$ and should 
            give a DLEQ proof that they have decrypted correctly, which 
            additionally requires $2\mathbb{E}_x$; Also the re-constructor 
            should verify DLEQ proofs of correct share decryption from $n-t$ 
            honest shareholders requiring them to do $4(n-t)\mathbb{E}_{x}$. 
            In total, each re-constructor requires $[3+4(n-t)]t\mathbb{E}_{x}$.
        \end{itemize}
    \item Using PPPVSS in randomness beacon protocol, a dealer(as a part of \textbf{commit}) requires to do $(n+l)[\mathbb{E}_x+\mathbb{P}_e]$ and $(l-1)\mathbb{M}_G$ to compute $\{y_i\}_{i=0}^{n}$. For generating the proof that $y_i$'s are valid encryptions of the secret shares and also $y_0$ is a commitment of the $l$ secrets, the dealer should do $(n+l)[\mathbb{E}_x+\mathbb{P}_e]$ which internally requires additional $(l-1)\mathbb{M}_G$. In total, a dealer has to do $2\left[(n+l)[\mathbb{E}_x+\mathbb{P}_e]+(l-1)\mathbb{M}_{G}\right]$.
    \begin{itemize}
        \item In \textbf{Reveal}, a verifier should do $(n+l)(2\mathbb{E}_x+\mathbb{P}_e)$ and $(l-1)\mathbb{M}_G$ for each proof. In total, a verifier has to do $(n-1)(n+l)[2\mathbb{E}_x+\mathbb{P}_e]+(n-1)(l-1)\mathbb{M_G}$.
        \begin{itemize}
            \item In \textbf{Robust case} with $t$ malicious parties not opening the secret polynomials, a verifier should do $l\mathbb{E}_x+(l-1)\mathbb{M}_G$ to verify each proof, so in total each verifier should do $(n-t-1)[l\mathbb{E}_x+(l-1)\mathbb{M}_G]$.
            \item In \textbf{Honest case} where everyone is honest, a verifier will do $(n-1)l(\mathbb{E}_x+\mathbb{M}_G)$.
        \end{itemize}
        \item The computational complexity of each re-constructor in \textbf{Recovery} phase is exactly same as in the case of ALBATROSS.
    \end{itemize}
\end{itemize}

\subsection{Computational Cost analysis}
The dealer has to do a bit more work in the case of our protocol in contrast to ALBATROSS, more explicitly, 
they have to compute $\ell$ more group exponentiations and polynomial evaluations. But as a consequence, 
we decrease computational cost in the \textit{Reveal} phase whenever $l < \frac{n(n-t-1)}{2(n-1)}$, roughly 
speaking, if the number of secrets are less than half of the honest parties then we always perform better in 
terms of computation when compared to the ALBATROSS.


\section{Communication Complexity}
\begin{table}[H]
\centering
\begin{tabular}{|p{3cm}|p{4cm}|p{3.5cm}|p{4cm}|p{1cm}|}
\hline
\textbf{Protocol}     & \textbf{Commit} \textit{(by Dealer)} & \textbf{Reveal} \textit{(by Dealer)} & \textbf{Recovery}  \textit{(by shareholder)}                                                      \\ \hline
\textbf{ALBATROSS}   & $nG+(t+l)\mathbb{Z}_q$ & $(t+l)\mathbb{Z}_q$ & $1G+1\mathbb{Z}_q+1R_o$ \\ \hline
\textbf{with PPPVSS}    & $(n+1)G+(t+l)\mathbb{Z}_q$ & $l\mathbb{Z}_q$ & $1G+1\mathbb{Z}_q+1R_o$ \\ \hline

\end{tabular}
\caption{Communication cost of dealer and (each) shareholder, $R_o$ being the random oracle, $G = $group of order $q$ and $\mathbb{Z}_q =$ modular group of order $q$, where $q$ is a large prime}
\label{tab:dealer_comm}
\end{table}

See table \ref{tab:dealer_comm} for an overview.
\begin{itemize}
    \item In ALBATROSS, a dealer (as a part of \textbf{commit}) should send $n$ group elements as commitments, $t+l$ elements in $\mathbb{Z}/q\mathbb{Z}$ that defines the polynomial used in the ZKP and $1$ extra element in $\mathbb{Z}/q\mathbb{Z}$ from RO. 
    \begin{itemize}
        \item In \textbf{Reveal}, an honest dealer would broadcast $t+l$ coefficients in $\mathbb{Z}/q\mathbb{Z}$ concerning the secret polynomial.
        \item If some party has not revealed their polynomial, then in \textbf{Recovery} phase a re-constructor using PVSS reconstruction protocol should broadcast $1$ element in group which is being the decrypted secret, for the proof of correct decryption, they have to broadcast $3$ more group elements along with a polynomial which requires $t+l$ coefficients in $\mathbb{Z}/q\mathbb{Z}$ and $1$ group element from RO.
    \end{itemize}
    \item Using PPPVSS in randomness beacon protocol, a dealer (as a part of \textbf{commit}) should send $n+1$ group elements as commitments, $t+l$ elements in $\mathbb{Z}/q\mathbb{Z}$ that defines the polynomial used in the ZKP and $1$ extra element in $\mathbb{Z}/q\mathbb{Z}$ from RO.
    \begin{itemize}
        \item In \textbf{Reveal}, an honest dealer would broadcast $l$ elements in $\mathbb{Z}_q$ concerning the exponents to construct the secret.
        \item If some part has not revealed their secrets, then the communication cost of each re-constructor is exactly same as in the case of ALBATROSS.
    \end{itemize}
\end{itemize}

\subsection{Communication Cost analysis}
The best to offer from our randomness beacon protocol is the communication cost. Though the dealer has to communicate only 
one extra group element compared to ALBATROSS in the commit phase, as a consequence for a fixed number of secrets the 
dealers' communication cost is constant as opposed to linear in number of corrupted parties in ALBATROSS. 

%%% Local Variables: 
%%% mode: latex
%%% TeX-master: "thesis"
%%% End: 
