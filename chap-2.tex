\chapter{More efficient PVSS and PPVSS}
\label{cha:2}
This chapter introduces a new PVSS scheme that is more efficient in terms of verification time 
than the state-of-the-art PVSS $\Pi_{S}$ \cite{cryptoeprint:2023/1669}. We noticed that the 
PVSS version where the secret is $g^{f_0}$ can never be secure against a malicious dealer who 
has access to unbounded computational power. This is because a computationally unbounded 
dealer can always cheat shareholders by giving them a fake proof of knowledge about the 
public-key encryptions of the secret.\par 

Due to this reason, we can use ZK AoK proof systems to improve the overall efficiency of 
PVSS schemes. In the remainder of this chapter, we will introduce a new NIZK AoK proof system 
and use it to construct a new PVSS scheme which is based on $\Pi_S$. As a consequence, 
we will also give a new PPVSS scheme.

\section{NIZK AoK for Polynomial DL}
\label{sec:AoK_R_pdl}
Consider the set $\{g_i\}_{i=1}^n$ containing $n$ distinct generators (obtained as CRS) of 
the prime order $q$ cyclic group $\mathcal{G}$ different from $g$, where $n<\varphi(q)$ and $\varphi(q)$ is 
the total number of distinct generators of the cyclic group $\mathbb{G}$. Consider the following relation for 
some polynomial $f\in\mathbb{Z}_q[X]_{t}$ with $0\leq t< n$:
\begin{align}\label{eq:relation_mod_PDL_ultimate}
  R_{PDL}^{mod} = \{(g_1,\dots,g_n,g,x_1,\dots,x_n,F_0,&\dots,F_n),f(X) : F_0=\prod_{i=1}^{n}g_i^{f(x_i)},\\\nonumber &F_i=g^{f(x_i)}, 1\leq i\leq n\},
\end{align}
where all $x_i$'s are distinct in $\mathbb{Z}_q$. The $R_{PDL}^{mod}$ is inspired from $R_{PDL}$ 
\cite{cryptoeprint:2023/1669} recalled in subsection \ref{subsec:polynomial-dl}. As a consequence, 
it is based on the following \textit{modified} PDL problem.

\begin{definition}[modified PDL problem]
  Let $g,g_1,\dots,g_n$ be distinct group generators of $\mathbb{G}$ of order $q$. 
  Given $\mathbb{F}\in\mathbb{G}$ and distinct elements $x_1,\dots,x_n$ in $\mathbb{Z}_q$, find 
  a polynomial $f(X)\in\mathbb{Z}_q[X]_t$ of at most degree $t$ such that 
  $F_0=\prod_{i=1}^{n}g_i^{f(x_i)}$ and $F_i=g^{f(x_i)}$ for $1\leq i\leq n$ where $0\leq t<n$.\par

  In other words, an algorithm $\mathcal{A}$ has advantage $\epsilon$ in solving the modified PDL problem in 
  $\mathbb{G}$ if:
  \begin{align*}
    \Pr[\mathcal{A}(x_1,\dots,x_n,g,g_1,\dots,g_n,\prod_{i=1}^{n}g_i^{f(x_i)},g^{f(x_1)},\dots,g^{f(x_n)})=f(X)]\geq \epsilon,
  \end{align*}
  where $f$ is at most a $t$ degree polynomial in $\mathbb{Z}_q[X]$ with $0\leq t<n$. The 
  probability is taken over distinct $x_i$'s in $\mathbb{Z}_q$, and the random with distinct 
  generators $g,g_1,\dots,g_n$ of $\mathbb{G}$.
\end{definition}

Consider the CRS dependent language of $R_{PDL}^{mod}$ as follows:
\begin{align*}
  L_{g_1,\dots,g_n}^{mod-PDL} = \{(g,x_1,\dots,x_n,&F_0,\dots,F_n) :\\ &\exists f\in\mathbb{Z}_q[X]_t, [(g_1,\dots,g_n,g,x_1,\dots,x_n,F_0,\dots,F_n),f(X)]\in R_{PDL}^{mod}\}
\end{align*}
We now present our NIZK AoK proof system $\pi_{PDL}^{mod}$ for the relation $R_{PDL}^{mod}$ in the 
figure 

%%% Local Variables: 
%%% mode: latex
%%% TeX-master: "thesis"
%%% End: 
