\chapter{Packed Pre-Constructed Publicly Verifiable Secret Sharing}
\label{cha:2}
In most of the real time distributed applications, each participant runs as a dealer once to share their 
secrets with other participants. To retrieve back the secret, remaining participants need to open their 
shares and perform a bunch of operations to reconstruct the secret, which requires a lot of communication 
amongst the participants. During the whole time of secret reconstruction one could have asked to the dealer 
itself to reveal the secret which will save a lot of time and communication. This is the main motivation 
behind the work of \textit{Pre-Constructed Publicly Verifiable Secret Sharing} (PPVSS) \cite{cryptoeprint:2025/576}, 
where the authors have given the complete description of PPVSS and an example $\Lambda_{RO}$ to improve 
the original e-voting protocol proposed by Schoenmakers \cite{5581ccd9530540479539d21d1d39ae96}.\par

In this chapter we will introduce Packed Pre-Constructed Publicly Verifiable Secret Sharing (PPPVSS) which merely 
is an extension to the notion of PPVSS. As Packed Shamir Secret Sharing \ref{sec:packed-shamir} allows a 
dealer to share multiple secrets encoded in a single polynomial, we want to add this feature to PPVSS. In the 
subsequent sections we will give our definitions for PPPVSS and provide with two constructions based on packed secret sharing 
scheme \ref{sec:packed-shamir} along with the security guarantees our schemes can achieve.\par

\section{Definitions}
\label{sec:pppvss-definitions}

The following definition directly follows the definition of PPVSS in \cite{cryptoeprint:2025/576}.

\begin{definition}[Packed Pre-Constructed Publicly Verifiable Secret Sharing (PPPVSS)]
    A Packed PPVSS scheme should have four algorithms, namely, Initial, Share, Verify and Reconstruction whose 
    descriptions are as follows:
    \begin{itemize}
        \item \textbf{Initial} $(1^\lambda)\rightarrow(\{PK_i,SK_i\}_{i=1}^n\sqcup\{h_j\text{ or }(PK_j,SK_j)\}_{j=-(\ell-1)}^0)$: 
          When given $1^\lambda$, each party $P_i$ for $1\leq i\leq n$ registers their public key $PK_i$ in a 
          public ledger and withholds the corresponding secret key $sk_i$. Also, all parties and the dealer $D$ 
          agree on $\ell$ commitment keys or public keys whose secret keys are known to some target people. 
          Note that the message space of the public-key scheme is a subgroup of $(\mathbb{G},\times)$.
        \item \textbf{Share} $(n,t,\{f_j\}_{j=-(\ell-1)}^0,\{PK_i\}_{i=1}^n\bigsqcup\{h_j\text{ or }PK_j\}_{j=-(\ell-1)}^0)$\par
          $\rightarrow(\{y_i\}_{i=-(\ell-1)}^n,\pi_{PPPVSS})$:\par 
          It secret shares $\{f_j\}_{j=(\ell-1)}^0$ to obtain the shares $\{f_i\}_{i=1}^n$. For $1\leq i\leq n$, uses 
          the public key $PK_i$ to encrypt $f_i$ and obtain the encrypted share $y_i$. Now, it uses the commitment 
          key $h_j$ ( or public key $PK_j$) to commit(/encrypt) $f_j$ to obtain $y_j$ for $-(\ell-1)\leq j\leq0$. 
          In the next step, it uses a NIZK proof $\pi_{PPPVSS}$ protocol to prove that $\{y_j\}_{j=-(\ell-1)}^0$ is a set of valid 
          commitments(/encryptions) of $\ell$ secrets and $\{y_i\}_{i=1}^n$ has valid encryptions of the corresponding shares. Finally, it returns $(\{y_i\}_{i=-(\ell-1)}^n,\pi_{PPPVSS})$.\par
          \textit{\textbf{OR}}\par
          $\rightarrow(\{y_i\}_{i=0}^n,\pi_{PPPVSS})$:\par 
          It secret shares $\{f_j\}_{j=(\ell-1)}^0$ to obtain the shares $\{f_i\}_{i=1}^n$. For $1\leq i\leq n$, uses 
          the public key $PK_i$ to encrypt $f_i$ and obtain the encrypted share $y_i$. Now, it uses the commitment 
          key $h_j$ ( or public key $PK_j$) to commit(/encrypt) $f_j$ for $-(\ell-1)\leq j\leq0$ and performs the 
          group operation $\times$ to multiply all the commitments(/encryptions) to obtain $y_0$. 
          In the next step, it uses a NIZK proof $\pi_{PPPVSS}$ protocol to prove that $y_0$ is a valid well-formed commitment of $\ell$ 
          secrets and $\{y_i\}_{i=1}^n$ has valid encryptions of the corresponding shares. Finally, it returns $(\{y_i\}_{i=0}^n,\pi_{PPPVSS})$.
        \item \textbf{Verify} $(n,t,\ell,\{y_i\}_{i=-(\ell-1)}^n,\pi_{PPPVSS})\rightarrow$ \textbf{true/false}: 
          This algorithm(which can be performed by anyone) checks if the NIZK proof $\pi_{PPPVSS}$ is valid for 
          $\{y_{-(\ell-1)},\dots,y_0,\dots,y_n\}$ and then returns true, otherwise false.\par
          \textit{\textbf{OR}}\par
          \textbf{Verify} $(n,t,\ell,\{y_i\}_{i=0}^n,\pi_{PPPVSS})\rightarrow$ \textbf{true/false}: 
          This algorithm(which can be performed by anyone) checks if the NIZK proof $\pi_{PPPVSS}$ is valid for 
          $\{y_0,y_1,\dots,y_n\}$ and then returns true, otherwise false.
        \item \textbf{Reconstruct}: There are two approaches based on cooperation of the dealer $D$ and are as follows:
          \begin{itemize}
            \item \textbf{(Optimistic)} $Reconstruction^{opt}[\{h_j,f_j,r_j,y_j\}_{j=-(\ell-1)}^0\text{ or }\{PK_j,SK_j,y_j\}_{j=-(\ell-1)}^0]$\par$\rightarrow(\{f_j\}_{j=-(\ell-1)}^0\text{ or }false)$:
              \begin{itemize}
                \item Given the input a verifier checks if $\{y_j\}_{j=-(\ell01)}^0$ is a valid set of commitments of the $\ell$ secrets. 
                 If so, it returns $\{f_j\}_{j=-(\ell-1)}^0$; otherwise it returns $false$.
                \item Given the public keys with their corresponding secret keys,\par $\{(PK_j,SK_j)\}_{j=-(\ell-1)}^0$ 
                 a verifier checks if $\{y_j\}_{j=-(\ell-1)}^0$ is a valid set of encryptions of the $\ell$ secrets. 
                 If so, it returns $\{f_j\}_{j=-(\ell-1)}^0$; otherwise it returns $false$.
              \end{itemize}\par
              \textit{\textbf{OR}}\par
              \textbf{(Optimistic)} $Reconstruction^{opt}[(\{h_j,f_j,r_j\}_{j=-(\ell-1)}^0,y_0)\text{ or }(\{PK_j,SK_j\}_{j=-(\ell-1)}^0,y_0)]$\par$\rightarrow(\{f_j\}_{j=-(\ell-1)}^0\text{ or }false)$:
              \begin{itemize}
                \item Given the commitment keys $\{h_j\}_{j=-(\ell-1)}^0$, $y_0$ and $\ell$ secrets, $\{f_j,r_j\}_{j=-(\ell-1)}^0$, 
                 a verifier checks if $y_0$ is a valid well-formed commitment of the $\ell$ secrets constructed by taking the product of commitment of $\ell$ secrets. If so, it returns 
                 $\{f_j\}_{j=-(\ell-1)}^0$; otherwise it returns $false$.
                \item Given the public keys with their corresponding secret keys,\par $\{(PK_j,SK_j)\}_{j=-(\ell-1)}^0$ 
                 a verifier checks if $y_0$ is a valid well-formed commitment of the $\ell$ secrets constructed by taking the product of encryptions of $\ell$ secrets. 
                 If so, it returns $\{f_j\}_{j=-(\ell-1)}^0$; otherwise it returns $false$.
              \end{itemize}
            \item \textbf{(Pessimistic)} $Reconstruction^{pes}[\{y_i,SK_i\}_{i\in\mathcal{Q},|\mathcal{Q}|=t+\ell}]\rightarrow(\{f_j\}_{j=-(\ell-1)}^0\text{ or }false))$: 
              Given any $t+\ell$ encrypted shares along with corresponding secret keys, it outputs the secrets 
              $\{f_j\}_{j=-(\ell-1)^0}$ or $false$. This can be done in two phases as follows: 
                \begin{itemize}
                  \item \textbf{Decryption of the shares}, each party $P_i\in\mathcal{Q}$ decrypts $y_i$ to obtain 
                    $f_i$ using its secret key $SK_i$. Then it generates a NIZK proof $\pi_i^{Dec}$ which proves that 
                    $f_i$ is the correct decryption of $y_i$. Now, $P_i$ publishes $(f_i,\pi_i^{Dec})$.
                  \item \textbf{Share pooling}, a verifier $V$ (not necessarily from the shareholders) checks if proof 
                   $\pi_i^{Dec}$ is correct for each $P_i\in\mathcal{Q}$. If any check fails, then $V$ returns $false$; 
                   otherwise $V$ applies a reconstruction procedure to the set of valid shares, $\{f_i\}_{i\in\mathcal{Q},|\mathcal{Q}|=t+\ell}$, 
                   and returns $\{f_j\}_{j=-(\ell-1)}^0$.
                \end{itemize}
          \end{itemize}
    \end{itemize}
\end{definition}

We need following security guarantees (inspired from PPVSS \cite{cryptoeprint:2025/576}) for a PPPVSS to be 
called secure:

\begin{itemize}
  \item \textbf{Correctness}: If the dealer and parties follow the protocol, then the \textit{\textbf{Verify}} 
    algorithm returns \textit{\textbf{true}} and the \textit{\textbf{Reconstruct}} algorithm returns $\{f_j\}_{j=-(\ell-1)}^0$ 
    irrespective of which approach. For any integer $n\geq t+\ell$ with $t>0$, a PPPVSS is said 
    to be correct for\par 
    $\left(\{PK_i,SK_i\}_{i=1}^n\sqcup\{h_j\text{ or }(PK_j,SK_j)\}_{j=-(\ell-1)}^0\right)\leftarrow$\textbf{Initial} $(1^\lambda)$ when 
    it satisfies either of the two formal definitions based on the output of the \textit{Share} algorithm 
    and they are as follows: 
    \begin{itemize}
      \item When \textit{Share} algorithm outputs $\ell$ commitments(/encryptions) of the secrets, then
        \begin{align*}
          Pr\begin{bmatrix}
            (\{y_i\}_{i=-(\ell-1)}^n,\pi_{PPPVSS})&\leftarrow \textit{\textbf{Share}} \big(n,t,\{f_j\}_{j=-(\ell-1)}^0,\{PK_i\}_{i=1}^n\bigsqcup\\
            &\{h_j\text{ or }PK_j\}_{j=-(l-1)}^0\big):\\
            \textit{\textbf{true}}&\leftarrow \textit{\textbf{Verify}} (n,t,\ell,\{y_i\}_{i=-(\ell-1)}^n,\pi_{PPPVSS})
          \end{bmatrix} = 1,
        \end{align*}
    
        \begin{align*}
          Pr\begin{bmatrix}
            (\{y_i\}_{i=-(\ell-1)}^n,\pi_{PPPVSS})&\leftarrow \textit{\textbf{Share}} \big(n,t,\{f_j\}_{j=-(\ell-1)}^0,\{PK_i\}_{i=1}^n\bigsqcup\\
            &\{h_j\text{ or }PK_j\}_{j=-(l-1)}^0\big),\\
            \{f_j^{'}\}_{j=-(\ell-1)}^0&\leftarrow Reconstruction^{opt}[\{h_j,f_j,r_j,y_j\}_{j=-(\ell-1)}^0\\
            &\textit{ or }\{PK_j,SK_j,y_j\}_{j=-(\ell-1)}^0]\bigvee\\
            \{f_j^{'}\}_{j=-(\ell-1)}^0&\leftarrow Reconstruction^{pes}[\{y_i,SK_i\}_{i\in\mathcal{Q},|\mathcal{Q}|=t+\ell}]:\\
            &f_j^{'}=f_j,-(\ell-1)\leq j\leq 0
          \end{bmatrix} = 1,
        \end{align*}
      \item When \textit{Share} algorithm outputs a single commitment corresponding to the $\ell$ secrets, then
        \begin{align*}
          Pr\begin{bmatrix}
            (\{y_i\}_{i=0}^n,\pi_{PPPVSS})&\leftarrow \textit{\textbf{Share}} \big(n,t,\{f_j\}_{j=-(\ell-1)}^0,\{PK_i\}_{i=1}^n\bigsqcup\\
            &\{h_j\text{ or }PK_j\}_{j=-(l-1)}^0\big):\\
            \textit{\textbf{true}}&\leftarrow \textit{\textbf{Verify}} (n,t,\ell,\{y_i\}_{i=0}^n,\pi_{PPPVSS})
          \end{bmatrix} = 1,
        \end{align*}
    
        \begin{align*}
          Pr\begin{bmatrix}
            (\{y_i\}_{i=0}^n,\pi_{PPPVSS})&\leftarrow \textit{\textbf{Share}} \big(n,t,\{f_j\}_{j=-(\ell-1)}^0,\{PK_i\}_{i=1}^n\bigsqcup\\
            &\{h_j\text{ or }PK_j\}_{j=-(l-1)}^0\big),\\
            \{f_j^{'}\}_{j=-(\ell-1)}^0&\leftarrow Reconstruction^{opt}[(\{h_j,f_j,r_j\}_{j=-(\ell-1)}^0,y_0)\\
            &\textit{ or }(\{PK_j,SK_j\}_{j=-(\ell-1)}^0,y_0)]\bigvee\\
            \{f_j^{'}\}_{j=-(\ell-1)}^0&\leftarrow Reconstruction^{pes}[\{y_i,SK_i\}_{i\in\mathcal{Q},|\mathcal{Q}|=t+\ell}]:\\
            &f_j^{'}=f_j,-(\ell-1)\leq j\leq 0
          \end{bmatrix} = 1,
        \end{align*}
    \end{itemize}
  \item \textbf{Verifiability}: If \textit{\textbf{Verify}} returns \textit{\textbf{true}}, then with high 
    probability $\{y_i\}_{i=1}^n$ are valid encryptions of shares of some $\ell$ secrets, and $y_0$ is a 
    valid well-formed commitment of the same $\ell$ secrets constructed by taking a product of valid commitments (/ciphers) of 
    $\ell$ secrets under $\{h_j\textit{ or }PK_j\}_{j=-(\ell-1)}^0$. Moreover, if the check in \textit{\textbf{(Pessimistic)}} $Reconstruction^{pes}$ 
    passes then the decrypted values are indeed shares of the secret distributed by the dealer. Alternatively, 
    if the check in \textit{\textbf{(Optimistic)}} $Reconstruct^{opt}$ passes then the input values $\{f_j\}_{j=-(\ell-1)}^0$ 
    are the actual secrets of whom the shares are encrypted. More formally, given $\lambda$, for any integers 
    $n\geq 2t+\ell$, $\ell\geq 1$ and $t\geq 0$, a PPPVSS is said to be verifiable if for any $\mathcal{PPT}$ adversary $\mathcal{A}$, 
    we have:
    \begin{align*}
      Pr\begin{bmatrix}
        \big(\{PK_i,SK_i\}_{i=1}^n\sqcup&\{h_j\text{ or }(PK_j,SK_j)\}_{j=-(\ell-1)}^0\big)\leftarrow\textbf{ Initial } (1^\lambda),\\
        (\{y_i\}_{i=0}^n,\pi_{PPPVSS})&\leftarrow \mathcal{A} \big(n,t,\{f_j\}_{j=-(\ell-1)}^0,\{PK_i\}_{i=1}^n\bigsqcup\\
        &\{h_j\text{ or }PK_j\}_{j=-(l-1)}^0\big),\\
        \{f_j^{'}\}_{j=-(\ell-1)}^0&\leftarrow Reconstruction^{opt}[(\{h_j,f_j,r_j\}_{j=-(\ell-1)}^0,y_0)\\
        &\textit{ or }(\{PK_j,SK_j\}_{j=-(\ell-1)}^0,y_0)]\bigvee\\
        \{f_j^{'}\}_{j=-(\ell-1)}^0&\leftarrow Reconstruction^{pes}[\{y_i,SK_i\}_{i\in\mathcal{Q},|\mathcal{Q}|=t+\ell}]:\\
        \textit{\textbf{true}}&\leftarrow \textit{\textbf{Verify}} (n,t,\ell,\{y_i\}_{i=0}^n,\pi_{PPPVSS})\\
        &\bigwedge \{f_j^{'}\}_{j=-(\ell-1)}^0\neq \{f_j\}_{j=-(\ell-1)}^0
      \end{bmatrix} \leq negl(\lambda),
    \end{align*}
    where $\mathcal{Q}$ is the set of honest parties.
  \item \textbf{IND1-Secrecy (Indistinguishability of Secrets)}: Before reconstruction phase, any amount of public 
    information along with secret keys of at most $t$ parties excluding $\{SK_j\}_{j=-(\ell-1)}^0$ will give 
    absolutely no information about the secrets $\{f_j\}_{j=-(\ell-1)}^0$. More formally, for integers $n>1$ and 
    $t+\ell-1<n$, the PPPVSS is said to satisfy \textit{IND1-Secrecy} if for any $\mathcal{PPT}$ adversary $\mathcal{A}$ 
    corrupting at most $t$ parties, excluding the owners of $\{SK_j\}_{j=-(\ell-1)}^0$, has negligible 
    advantage in the following game \cite{cryptoeprint:2025/576} played against a challenger.
    \begin{itemize}
      \item The challenger runs \textit{\textbf{Initial}} $(1^\lambda)$ of PPPVSS to obtain 
        $\{PK_i,SK_i\}_{i=1}^n\sqcup\{h_j\text{ or }(PK_j,SK_j)\}_{j=-(\ell-1)}^0$ and sends all public 
        information along with secret information of all corrupted parties to $\mathcal{A}$.
      \item The challenger chooses two set of secrets, $s_0=\{f_j\}_{j=-(\ell-1)}^0$ and 
        $s_1=\{f_j^{'}\}_{j=-(\ell-1)}^0$ at random in the space of secrets. Furthermore, it chooses $b\in\{0,1\}$ 
        uniformly at random and runs \par 
        \textit{\textbf{Share}} $(n,t,s_0,\{PK_i\}_{i=1}^n\bigsqcup\{h_j\text{ or }PK_j\}_{j=-(\ell-1)}^0)$ 
        algorithm of the PPPVSS scheme and obtains $(\{y_i\}_{i=0}^n,\pi_{PPPVSS})$. Finally, it sends 
        all public information generated in \textit{Share} phase together with $s_b$.
      \item $\mathcal{A}$ guesses a bit $b'\in\{0,1\}$.
    \end{itemize}
    The advantage of $\mathcal{A}$ is defined to be $|Pr[b'=b]-\frac{1}{2}|$.
\end{itemize}

It can be seen that PPPVSS is a natural extension of PPVSS, where we secret share possibly \textit{more than one} 
secret in contrast to \textit{only one} secret in PPVSS. But one has to be careful when obtaining the commitment 
of the secrets which is obtained via multiplication of encryptions (/commitments) of actual secrets. 
More precisely, $y_0$ should be well formed, i.e., a dealer should not be able to cheat to obtain $y_0$ by 
taking multiplications of random group elements, it precisely has to be multiplication of encryptions 
(/commitments) of the actual secrets.\par 

A general construction of Shamir based PPVSS is given in \cite{cryptoeprint:2025/576}, and we extend this idea 
to give a Packed Shamir based PPPVSS which is outlined in the figure \ref{fig:packed-shamir-PPPVSS}. 
In next section, we will give a practical PPPVSS scheme.

\begin{figure}[ht]
    \centering
    \resizebox{\textwidth}{!}{ % Further reduced the scaling factor to fit the content
    \begin{tcolorbox}[title=\textbf{PPPVSS based on Packed Shamir secret sharing \ref{sec:packed-shamir}}, width=1.2\textwidth, colframe=blue!75!black, colback=blue!10, sharp corners]
        
        \textbf{Initialization:}
            All parties $\{P_i\}_{i=1}^n$ and dealer $D$ agree on the prime field $\mathbb{Z}_q$. Also, each party 
            $P_i$ registers their public key $PK_i$ in the public ledger and all agree on a set of commitment 
            keys or public keys, $\{h_j\text{ or }PK_j\}_{j=-(\ell-1)}^0$, whose secret keys are known to target entities with 
            the cipher text space being a group $(\mathcal{G},\times)$. More importantly, all entities agree 
            on two NIZK proof protocols, one for the dealer during sharing phase $\pi_{PPPVSS}^{share}$ and the 
            other for Pessimistic reconstruction phase $\pi_{PPPVSS}^{pes}$.

        \vspace{0.5em}
        \textbf{Share:}
        \begin{itemize}
            \item Dealer $D$ samples a $t+\ell$-degree polynomial $f\in\mathbb{Z}_q[X]$ uniformly at random and 
              sets $\{f(j)=f_j\}_{j=-(\ell-1)}^0$ as the set of all secrets.
            \item For each $1\leq i\leq n$, $D$ computes $f(i)=f_i$ and and encrypts it with $PK_i$ to obtain 
              $y_i=Enc(PK_i,f_i)$. $D$ also encrypts(/commits) to the secrets $\{f_j\}_{j=-(\ell-1)}^0$ using the encryption(/commitment) 
              keys $\{h_j\text{ or }PK_j\}$ and multiplies them together to obtain $y_0$. 
            \item $D$ uses the agreed upon proof system $\pi_{PPPVSS}^{share}$ to prove that they have encrypted the shares and 
              committed to the $\ell$ secrets corresponding to the correct polynomial that evaluates to $\ell$ secrets.
            \item $D$ broadcasts $\{y_0,\dots,y_n,\pi_{PPPVSS}^{share}\}$.
        \end{itemize}
        
        \vspace{0.5em}
        \textbf{Verification:}
            Given public keys $\{PK_i\}_{i=1}^n$ and commitment(/public) keys $\{h_j,PK_j\}_{j=-(\ell-1)}^0$, 
            any entity can check $\pi_{PPPVSS}^{share}$ to verify the correctness of the encrypted shares $\{y_i\}_{i=1}^n$ and 
            $y_0$ being the commitment of the $\ell$ secrets, and outputs \textbf{true} or \textbf{false} accordingly.

        \vspace{0.5em}
        \textbf{Reconstruction:}
            Similar to \cite{cryptoeprint:2025/576}, there are two approaches to reconstruct the secrets 
            $\{f_j\}_{j=-(\ell-1)}^0$ based on the cooperation of the dealer $D$ which are as follows:
            \begin{itemize}
                \item \textbf{Optimistic Reconstruction:} $D$ publishes $\{f_j\}_{j=-(\ell-1)}^0$, then any verifier (not necessarily shareholders) 
                when given $\{h_j\text{ or }PK_j\}_{j=-(\ell-1)}^0,y_0$ can check if $y_0$ is the valid product of the commitment(/encryptions) 
                of the secrets $\{f_j\}_{j=-(\ell-1)}^0$ and returns \textbf{false} if the check fails; otherwise returns 
                $\{f_j\}_{j=-(\ell-1)}^0$.
                \item \textbf{Pessimistic Reconstruction:} If $D$ refuses to reveal $\{f_j\}_{j=-(\ell-1)}^0$, then any set 
                $\mathcal{Q}$ consisting at least $t+\ell$ shareholders will do the following:
                \begin{itemize}
                    \item Each party $P_i\in\mathcal{Q}$ decrypts their share $y_i$ using their private key $SK_i$ 
                      corresponding to $PK_i$ to obtain $f_i$ and then they publish $f_i$ 
                      along with proof $\pi_{PPPVSS}^{pes}$ which proves that the $f_i$ is the correct 
                      decryption of $y_i$.
                    \item If $\mathcal{Q}$ consists at least $t+\ell$ honest parties, they can use the 
                    lagrange interpolation to compute the secrets $\{f_j\}_{j=-(\ell-1)}^0$ as follows:
                    \begin{align*}
                        f_j &= \sum_{i\in \mathcal{Q}} f_i \left[\prod_{k\in \mathcal{Q}, k\neq i}\frac{-k-j}{i-j}\right] &&, j\in\{-(\ell-1),\dots,0\} \\
                    \end{align*}
                    where $\lambda_i=\prod_{k\neq i}\frac{k}{k-i}$ are lagrange coefficients.
                \end{itemize}
            \end{itemize}
    \end{tcolorbox}
    }
    \caption[PPPVSS Scheme]{PPPVSS based on Packed Shamir secret sharing, version 2}
    \label{fig:packed-shamir-PPPVSS}
\end{figure}

\section{A NIZK AoK for \textit{modified}-PDL}
\label{sec:aok_polynomial_dl}
Consider the set $\{g_j\}_{j=-(\ell-1)}^0$ containing $\ell$ distinct generators chosen uniformly at random of 
the prime order $q$ cyclic group $\mathcal{G}$ different from $g$, where $\ell<\varphi(q)$ and $\varphi(q)$ are 
the total number of distinct generators of the cyclic group $\mathbb{G}$. Consider the following relation for 
some polynomial $f\in\mathbb{Z}_q[X]_{t+\ell-1}$ with $t\leq n$:
\begin{align}\label{eq:relation_mod_PDL}
  R_{PDL}^{mod} = \{(g,\{g_j,x_j\}_{j=-(\ell-1)}^0,&\{x_i\}_{i=1}^n,\{F_i\}_{i=0}^n),f(X) :\nonumber\\
   &F_0=\prod_{j=-(\ell-1)}^{0}g_j^{f(x_j)}, F_i=g^{f(x_i)}, 1\leq i\leq n\},
\end{align}
where all $x_i$'s are distinct in $\mathbb{Z}_q$. The $R_{PDL}^{mod}$ is based on \textit{modified}-PDL (inspired from the PDL in \cite{cryptoeprint:2023/1669}) 
which is defined in the following.

\begin{definition}[\textit{modified}- Polynomial Discrete Logarithm]
  Let\par 
  $\{g_j : g_j\neq g\}_{j=-(\ell)-1}^0$ be a set of distinct generators for the prime order $q$ cyclic 
  group $\mathbb{G}$ generated by $g$. Given $F_0,\dots,F_n$ and distinct elements $x_{-(\ell-1)},\dots,x_0,\dots,x_n$ in 
  $\mathbb{Z}_q$, find a polynomial $f\in\mathbb{Z}_q[X]$ with degree at most $t+\ell-1$, where $F_0=\prod_{j=-(\ell-1)}^{0}g_j^{f(x_j)}$, 
  $F_i=g^{f(x_i)}$ for $1\leq i\leq n$ and $t\leq n$.\par

  In other words, an algorithm $\mathcal{A}$ is said to have an advantage $\epsilon$ in solving \textit{modified}-PDL if 
  \begin{align*}
    Pr[\mathcal{A}(x_{-(\ell-1)},\dots,x_0,\dots,x_n,g,\{g_j\}_{j=-(\ell-1)}^0,\prod_{j=-(\ell-1)}^{0}g_j^{f(x_j)},g^{f(x_1)},\dots,g^{f(x_n)})]\geq\epsilon,
  \end{align*}
  where $f\in\mathbb{Z}_q[X]$ is at most a $t+\ell-1$ degree polynomial with $t\leq n$ and the probability is over 
  distinct generators $g,g_{-(\ell-1)},\dots,g_0$ of $\mathbb{G}$ chosen at random and distinct $x_{-(\ell-1)},\dots,x_0,\dots,x_n$ 
  elements in $\mathbb{Z}_q$.
\end{definition}

We now will present a new NIZK AoK for the aforementioned relation $R_{PDL}^{mod}$, which is inspired from \ref{fig:polynomial-dl}, 
in the figure \ref{fig:mod-polynomial-dl}.

\begin{theorem}[A NIZK Argurement of Knowledge for $R_{PDL}^{mod}$]\label{th:modified_PDL security}
  Let\par $(g,\{g_j,x_j\}_{j=-(\ell-1)}^0,\{x_i\}_{i=1}^n,\{F_i\}_{i=0}^n)\in L_{PDL}^{mod}$ be a statement 
  with its corresponding witness being $f\in\mathbb{Z}_q[X]_{\leq t+\ell-1}$ where $L_{PDL}^{mod}$ is the 
  language defined by the relation $R_{PDL}^{mod}$. Assuming \textit{modified}-PDL is computationally hard, for 
  $0\leq t\leq n$, the protocol $\pi_{PDL}^{mod}$ (described in figure \ref{fig:mod-polynomial-dl}) is a 
  NIZK AoK for $R_{PDL}^{mod}$ in  the $RO$ model.
\end{theorem}
\begin{proof}
  The corresponding proof is similar to the proof of theorem \ref{th:PDL security} given in \cite{cryptoeprint:2023/1669}. 
  Formally, we will prove the security of the interactive setting (i.e., without $RO$ being used) and then 
  using Fiat-Shamir transform it can be extended for the non-interactive setting in the $RO$ model. 

  \begin{itemize}
    \item \textit{\textbf{Correctness}}: If both prover and verifier are honest, then for $1\leq i\leq n$ 
      we have 
      \begin{align}\label{eq:i_verify}
        g^{z(i)}&=g^{r(i)+df(i)}\nonumber\\
        &=g^{r(i)}(g^{f(i)})^d\nonumber\\
        &=\Gamma_i F_i^d,
      \end{align}
      and 
      \begin{align}\label{eq:0_verify}
        \prod_{j=-(\ell-1)}^{0}g_j^{z(j)}&=\prod_{j=-(\ell-1)}^{0}g_j^{r(j)+df(j)}\nonumber\\
        &=(\prod_{j=-(\ell-1)}^{0}g_j^{r(j)})(\prod_{j=-(\ell-1)}^{0}g_j^{f(j)})^d\nonumber\\
        &=\Gamma_0 F_0^d.
      \end{align}
      The aforementioned equations imply that verification returns \textit{true} and honest verifier accepts the 
      honest prover.
    \item \textit{\textbf{Special Soundness}}: Let $(\Gamma_0,\dots,\Gamma_n,d,z(X))$, $(\Gamma_0,\dots,\Gamma_n,d',z'(X))$ 
      be two acceptable transcripts where response polynomials will differ as a consequence of different challenge 
      values. For $1\leq i\leq n$ we have the following from equation \ref{eq:i_verify}:
      \begin{align*}
        g^{z(x_i)}=\Gamma_i F_i^d,g^{z'(x_i)}=\Gamma_i F_i^{d'};
      \end{align*}
      which implies
      \begin{align}\label{eq:i_Equality}
        g^{z(x_i)-z'(x_i)}=F_i^{d-d'} \iff g^{\frac{z(x_i)-z'(x_i)}{d-d'}}=F_i,
      \end{align}
      [\textit{if and only if because $d-d'$ is always invertible in $\mathbb{Z}_q$ for $d\neq d'\pmod{q}$}]. Similarly, 
      we have the following from equation \ref{eq:0_verify}:
      \begin{align*}
        \prod_{j=-(\ell-1)}^{0}g_j^{z(x_j)}=\Gamma_0 F_0^d,\prod_{j=-(\ell-1)}^{0}g_j^{z'(x_j)}=\Gamma_0 F_0^{d'};
      \end{align*}
      implying
      \begin{align}\label{eq:0_Equality}
        \prod_{j=-(\ell-1)}^{0}g_j^{z(x_j)-z'(x_j)}=F_0^{d-d'} \iff \prod_{j=-(\ell-1)}^{0}g_j^{\frac{z(x_j)-z'(x_j)}{d-d'}}=F_0.
      \end{align}
      Hence, equations \ref{eq:i_Equality} and \ref{eq:0_Equality} imply that $f_i=\frac{z(x_i)-z'(x_i)}{d-d'}$ for 
      $-(\ell-1)\leq i\leq n$ when all $n+\ell$ checks pass. Moreover, as $z(X)$ is at most a $t+\ell-1$ degree 
      polynomial in $\mathbb{Z}_q[X]$, an extractor $\mathcal{E}$ can construct the unique $t+\ell-1$-degree 
      polynomial $f\in\mathbb{Z}_q[X]$, being the witness (resp. solution) for $R_{PDL}^{mod}$ relation (resp. \textit{modified}-PDL problem), 
      from any $t+\ell$ evaluation points in $\{f_i\}_{i=-(\ell-1)}^n$ whenever $n\geq t$.
    \item \textit{\textbf{Honest Verifier Zero Knowledge (HVZK)}}: Given the statement\par $(g,\{g_j,x_j\}_{j=-(\ell-1)}^0,\{x_i\}_{i=1}^n,\{F_i\}_{i=0}^n)\in L_{PDL}^{mod}$  
      and a challenge value $d$, a simulator $\mathcal{S}$ can choose a polynomial $z'\in\mathbb{Z}_q[X]_{\leq t+\ell-1}$ uniformly 
      at random and sets $\Gamma_i'=\frac{g^{z'(x_i)}}{F_i^d}$ for $1\leq i\leq n$ and $\Gamma_0'=\frac{\prod_{j=-(\ell-1)}^{0}g_j^{z(x_j)}}{F_0^d}$. 
      Now, $\mathcal{S}$ returns $(\{\Gamma_i\}_{i=0}^n,z'(X))$ as the simulated proof. As $z(X)$ is a random 
      up to degree $t+\ell-1$-polynomial in $\mathbb{Z}_q[X]$ and $z'(X)$ is chosen uniformly at random, 
      the simulated proof of $\mathcal{S}$ is indistinguishable from the real one.
  \end{itemize}
  As the interactive scheme is public coin, satisfies \textit{completeness}, (computational) \textit{Special Soundness} 
  and (computational) \textit{HVZK}, then in the random oracle ($RO$) model, using Fiat-Shamir transform \cite{10.1007/3-540-47721-7_12}, 
  it can be turned into a NIZK Argument of Knowledge for $R_{PDL}^{mod}$ (defined in equation \ref{eq:relation_mod_PDL}).
\end{proof}
\begin{figure}[ht]
    \centering
    \begin{tcolorbox}[title=$\pi_{PDL}^{mod}$, width=0.9\textwidth, colframe=blue!75!black, colback=blue!10, sharp corners]
        Let $(g,x_{-(\ell-1)},\dots,x_n,F_0,\dots,F_n)\in L_{g_{-(\ell-1)},\dots,g_n}^{mod-PDL}$ with its corresponding witness being $f\in\mathbb{Z}_q[X]_{\leq t+\ell-1}$ 
        where $L_{g_{-(\ell-1)},\dots,g_n}^{mod-PDL}$ is the CRS dependent language defined by 
        the relation $R_{mod-PDL}$ and $\mathcal{H}$ be a random oracle (RO).
        
        \vspace{0.5em}
        \textbf{Prover}
        \begin{itemize}
            \item Samples $r\in_{R}\mathbb{Z}_q[X]_{t+\ell-1}$ uniformly at random and sets 
                $\Gamma=\prod_{i=-(\ell-1)}^{n}g_i^{r(x_i)}$.
            \item Sets $d\leftarrow \mathcal{H}(F_0, \dots, F_n, \Gamma)$, where $\mathcal{H}$ is 
                an agreed upon Random Oracle (RO).
            \item Sets $z(X)\equiv r(X)+df(X) \pmod{q}$ and returns the proof(/transcript) $\pi:= (d,z(X))$.
        \end{itemize}
        
        \vspace{0.5em}
        \textbf{Verifier}
        \begin{itemize}
            \item First checks if $z$ is at most a $t+\ell-1$ degree polynomial in $\mathbb{Z}_q[X]$. If so, they proceed with the next step.
            \item Checks if $d\leftarrow \mathcal{H}(F_0, \dots, F_n,\frac{\prod_{i=-(\ell-1)}^{n}g_i^{z(x_i)}}{[\prod_{i=0}^{n}F_i]^d})$. 
            \item If first two steps are correct then they output \textbf{true}, otherwise \textbf{false}.
        \end{itemize}
    \end{tcolorbox}
    \caption{A NIZK AoK for the \textit{modified} Polynomial DL}
    \label{fig:mod-polynomial-dl}
\end{figure}


\section{Practical PPPVSS schemes}

The scheme we present in figures \ref{fig:initial-packed-shamir-PPPVSS-ro} and \ref{fig:packed-shamir-PPPVSS-ro} is a direct extension of $\Lambda_{RO}$ \cite{cryptoeprint:2025/576} 
(which originally is based on $\Pi_S$ \cite{cryptoeprint:2023/1669}). Fundamentally, we make use of the two NIZK 
PoK protocols, $\pi_{PDL}$ and $\pi_{DLEQ}$, to construct our PPPVSS. We give its security proofs in the following. 

\begin{figure}[ht]
    \centering
    \resizebox{\textwidth}{!}{ % Further reduced the scaling factor to fit the content
    \begin{tcolorbox}[title=\textbf{$\Lambda_{RO}^{'}$}, width=1.2\textwidth, colframe=blue!75!black, colback=blue!10, sharp corners]
        
        \textbf{Initialization:}
            All parties $\{P_i\}_{i=1}^n$ and dealer $D$ agree on the prime field $\mathbb{Z}_q$, a group 
            $(\mathbb{G},\times)$ of order $q$ with a generator $g$, random oracle $\mathcal{H}$. Also, each party 
            $P_i$ registers their public key $PK_i$ in the public ledger and all agree on a set of commitment 
            keys or public keys, $\{h_j\text{ or }PK_j\}_{j=-(\ell-1)}^0$, whose secret keys are known to target entities.

        \vspace{0.5em}
        \textbf{Share:}
        \begin{itemize}
            \item Dealer $D$ samples a $t+\ell$-degree polynomial $f\in\mathbb{Z}_q[X]$ uniformly at random and 
              sets $\{g^{f_j}:f_j=f(j)\}_{j=-(\ell-1)}^0$ as the set of all secrets.
            \item For each $1\leq i\leq n$, $D$ encrypts $f(i)=f_i$ with $PK_i$ to obtain 
              $y_i=PK_i^{f_i}$. $D$ also commits(/encrypts) to the secrets $\{f_j\}_{j=-(\ell-1)}^0$ using the encryption(/commitment) 
              keys $\{h_j\text{ or }PK_j\}_{j=-(\ell-1)}^0$ to obtain $\{y_j=h_j^{f_j}\}_{j=-(\ell-1)}^0$ (or $\{y_j=PK_j^{f_j}\}_{j=-(\ell-1)}^0$). 
            \item $D$ uses the PDL proof scheme \ref{subsec:polynomial-dl} to generate $\pi_{PDL}$ as follows:
            \begin{itemize}
               \item Samples a $t+\ell-$degree polynomial $r\in\mathbb{Z}_q[X]$ uniformly at random and 
               computes $\{c_j=h_j^{r(j)}\}_{j=-(\ell-1)}^0$(or $\{c_j=PK_j^{r(j)}\}_{j=-(\ell-1)}^0$) and 
               $c_i=PK_i^{r(i)}$ for $1\leq i\leq n$.
               \item Using $\mathcal{H}$, $d=\mathcal{H}(y_{-(\ell-1)}\dots,y_0,\dots,y_n,c_{-(\ell-1)},\dots,c_0,\dots,c_n)$ is computed.
               \item Sets $z(X)=r(X)+df(X)$, hence $\pi_{PDL}=(d,z(X))$ is obtained. 
            \end{itemize}
            \item $D$ broadcasts the encryptions of the shares along with the commitment (or product of the encryptions) of the secrets 
              with $\pi_{PDL}$ which proves the validity of the encrypted shares and committed (/encrypted)
              secrets, i.e., broadcasts $\{y_i\}_{i=-(\ell-1)}^n$ and $(d,z(X))$.
        \end{itemize}
        
        \vspace{0.5em}
        \textbf{Verification:}
            Given public keys $\{PK_i\}_{i=1}^n$ and commitment(/public) keys $\{h_j\text{ or }PK_j\}_{j=-(\ell-1)}^0$, any entity can check 
            $\pi_{PDL}$ to verify the correctness of the encrypted shares $\{y_i\}_{i=1}^n$ and $\{y_j\}_{j=-(\ell-1)}^0$ being the commitments(/encryptions) of the secrets. 
            They will output \textbf{true} or \textbf{false} based on the verification of the proof. The 
            procedure is outlined as follows:
        \begin{itemize}
            \item The entity checks if $z(X)$ is a $t+\ell-$degree polynomial or not.
            \item Checks if $d=\mathcal{H}(y_{-(\ell-1)},\dots,y_0,\dots,y_n,\frac{h_{-(\ell-1)}^{z(-(\ell-1))}}{y_{-(\ell-1)}^d},\dots,\frac{h_{0}^{z(0)}}{y_0^d},\frac{PK_1^{z(1)}}{y_1^d},\dots,\frac{PK_n^{z(n)}}{y_n^d})$ or 
            $d=\mathcal{H}(y_{-(\ell-1)},\dots,y_0,\dots,y_n,\frac{PK_{-(\ell-1)}^{z(-(\ell-1))}}{y_{-(\ell-1)}^d},\dots,\frac{PK_{0}^{z(0)}}{y_0^d},\frac{PK_1^{z(1)}}{y_1^d},\dots,\frac{PK_n^{z(n)}}{y_n^d})$ .
            \item Outputs \textbf{true} if both of the above checks are satisfied, otherwise \textbf{false}.
        \end{itemize}

        \vspace{0.5em}
        \textbf{Reconstruction:}
            Similar to \cite{cryptoeprint:2025/576}, there are two approaches to reconstruct the secrets 
            $\{g^{f_j}\}_{j=-(\ell-1)}^0$ based on the cooperation of the dealer $D$ which are as follows:
            \begin{itemize}
                \item \textbf{Optimistic Reconstruction:} $D$ publishes $\{f_j\}_{j=-(\ell-1)}^0$, then any verifier (not necessarily a shareholder) 
                when given $g,\{h_j\text{ or }PK_j\}_{j=-(\ell-1)}^0,y_{-(\ell-1)},\dots,y_0$ can check if $\{y_j=h_j^{f_j}\text{ or }PK_j^{f_j}\}_{j=-(\ell-1)}^0$ 
                and returns $\{g^{f_j}\}_{j=-(\ell-1)}^0$ if the check passes, if not they return \textbf{false}.
                \item \textbf{Pessimistic Reconstruction:} If $D$ refuses to reveal $\{f_j\}_{j=-(\ell-1)}^0$, then any set 
                $\mathcal{Q}$ consisting at least $t+\ell$ shareholders will do the following:
                \begin{itemize}
                    \item Each party $P_i\in\mathcal{Q}$ decrypts their share $y_i$ using their private key $SK_i$ 
                      corresponding to $PK_i$ to obtain $g^{f_i}$ and then they publish $g^{f_i}$ 
                      along with a DLEQ proof \ref{subsec:chaum-pedersen}, $\pi_{DLEQ}$ which proves that 
                      $g^{f_i}$ is the correct decryption of $y_i$.
                    \item If $\mathcal{Q}$ consists at least $t+\ell$ honest parties, they can use the 
                    lagrange interpolation to compute the secrets $\{g^{f_j}\}_{j=-(\ell-1)}^0$ as follows:
                    \begin{align*}
                        g^{f_j} &= \prod_{i\in\mathcal{Q}}(g^{f_i})^{\prod_{k\in\mathcal{Q},k\neq i}\frac{-k-j}{i-j}}= g^{\sum_{i\in\mathcal{Q}}f_i\prod_{k\in\mathcal{Q},k\neq i}\frac{-k-j}{i-j}}.\\
                    \end{align*}
                \end{itemize}
            \end{itemize}
    \end{tcolorbox}
    }
    \caption[PPPVSS]{$\Lambda_{RO}^{'}$, a packed version of $\Lambda_{RO}$ \cite{cryptoeprint:2025/576}}
    \label{fig:initial-packed-shamir-PPPVSS-ro}
\end{figure}

\begin{figure}[H]
    \centering
    \resizebox{\textwidth}{!}{ % Further reduced the scaling factor to fit the content
    \begin{tcolorbox}[title=\textbf{$\Lambda_{RO}^{packed}$}, width=1.2\textwidth, colframe=blue!75!black, colback=blue!10, sharp corners]
        
        \textbf{Initialization:}
            All parties $\{P_i\}_{i=1}^n$ and dealer $D$ agree on the prime field $\mathbb{Z}_q$, a group 
            $(\mathbb{G},\times)$ of order $q$ with a generator $g$, random oracle $\mathcal{H}$. Also, each party 
            $P_i$ registers their public key $PK_i$ in the public ledger and all agree on a set of commitment 
            keys or public keys, $\{h_j\text{ or }PK_j\}_{j=-(\ell-1)}^0$, whose secret keys are known to target entities.

        \vspace{0.5em}
        \textbf{Share:}
        \begin{itemize}
            \item Dealer $D$ samples a $t+\ell$-degree polynomial $f\in\mathbb{Z}_q[X]$ uniformly at random and 
              sets $\{g^{f_j}:f_j=f(j)\}_{j=-(\ell-1)}^0$ as the set of all secrets.
            \item For each $1\leq i\leq n$, $D$ encrypts $g^{f_i}$ ($f(i)=f_i$) using $PK_i$ to obtain 
              $y_i=PK_i^{f_i}$. $D$ also encrypts(/commits) to the secrets $\{g^{f_j}\}_{j=-(\ell-1)}^0$ using the encryption(/commitment) 
              keys $\{h_j\text{ or }PK_j\}_{j=-(\ell-1)}^0$ and multiplies them together to obtain $y_0=\prod_{j=-(\ell-1)}^0(h_j\text{ or }PK_j)^{f_j}$. 
            \item $D$ uses a variant of the modified-PDL proof scheme \ref{sec:aok_polynomial_dl} to generate $\pi_{share}$ as follows:
            \begin{itemize}
               \item Samples a $t+\ell-$degree polynomial $r\in\mathbb{Z}_q[X]$ uniformly at random and 
               computes $c_0=\prod_{j=-(\ell-1)}^0(h_j\text{ or }PK_j)^{r(j)}$ and 
               $c_i=PK_i^{r(i)}$ for $1\leq i\leq n$.
               \item Using $\mathcal{H}$, $d=\mathcal{H}(y_0,\dots,y_n,c_0,\dots,c_n)$ is computed.
               \item Sets $z(X)=r(X)+df(X)$, hence $\pi_{share}=(d,z(X))$ is obtained. 
            \end{itemize}
            \item $D$ broadcasts the encryptions of the shares along with the commitment of the secrets 
              with $\pi_{share}$ which proves the validity of the encrypted shares and committed 
              secret, i.e., broadcasts $\{y_i\}_{i=0}^n$ and $(d,z(X))$.
        \end{itemize}
        
        \vspace{0.5em}
        \textbf{Verification:}
            Given public keys $\{PK_i\}_{i=1}^n$ and commitment(/public) keys $\{h_j\text{ or }PK_j\}_{j=-(\ell-1)}^0$, any entity can check 
            $\pi_{share}$ to verify the correctness of the encrypted shares $\{y_i\}_{i=1}^n$ and $y_0$ being the commitment of the secrets. 
            They will output \textbf{true} or \textbf{false} based on the verification of the proof. The 
            procedure is outlined as follows:
        \begin{itemize}
            \item The entity checks if $z(X)$ is a $t+\ell-$degree polynomial or not.
            \item Checks if $d=\mathcal{H}(y_0,y_1,\dots,y_n,\frac{\prod_{j=-(\ell-1)}^{0}(h_j\text{ or }PK_j)^{z(j)}}{y_0^d},\frac{PK_1^{z(1)}}{y_1^d},\dots,\frac{PK_n^{z(n)}}{y_n^d})$.
            \item Outputs \textbf{true} if both of the above checks are satisfied, otherwise \textbf{false}.
        \end{itemize}

        \vspace{0.5em}
        \textbf{Reconstruction:}
            Similar to \cite{cryptoeprint:2025/576}, there are two approaches to reconstruct the secrets 
            $\{f_j\}_{j=-(\ell-1)}^0$ based on the cooperation of the dealer $D$ which are as follows:
            \begin{itemize}
                \item \textbf{Optimistic Reconstruction:} $D$ publishes $\{f_j\}_{j=-(\ell-1)}^0$, then any verifier (not necessarily shareholders) 
                when given $g,\{h_j\text{ or }PK_j\}_{j=-(\ell-1)}^0,y_0$ can check if $y_0=\prod_{j=-(\ell-1)}^0(h_j\text{ or }PK_j)^{f_j}$ 
                and then returns $\{g^{f_j}\}_{j=-(\ell-1)}^0$ when the check passes, if not they return \textbf{false}.
                \item \textbf{Pessimistic Reconstruction:} If $D$ refuses to reveal $\{f_j\}_{j=-(\ell-1)}^0$, then any set 
                $\mathcal{Q}$ consisting at least $t+\ell$ shareholders will do the following:
                \begin{itemize}
                    \item Each party $P_i\in\mathcal{Q}$ decrypts their share $y_i$ using their private key $SK_i$ 
                      corresponding to $PK_i$ to obtain $g^{f_i}$ and then they publish $g^{f_i}$ 
                      along with a DLEQ proof \ref{subsec:chaum-pedersen}, $\pi_{DLEQ}$ which proves that 
                      $g^{f_i}$ is the correct decryption of $y_i$.
                    \item If $\mathcal{Q}$ consists at least $t+\ell$ honest parties, they can use the 
                    lagrange interpolation to compute the secrets $\{g^{f_j}\}_{j=-(\ell-1)}^0$ as follows:
                    % \begin{align*}
                      $$ \textstyle  g^{f_j} = \prod_{i\in\mathcal{Q}}(g^{f_i})^{\prod_{k\in\mathcal{Q},k\neq i}\frac{-k-j}{i-j}}= g^{\sum_{i\in\mathcal{Q}}f_i\prod_{k\in\mathcal{Q},k\neq i}\frac{-k-j}{i-j}}. $$
                    % \end{align*}
                \end{itemize}
            \end{itemize}
    \end{tcolorbox}
    }
    \caption[PPPVSS]{$\Lambda_{RO}^{packed}$, an alternative packed version of $\Lambda_{RO}$ \cite{cryptoeprint:2025/576}}
    \label{fig:packed-shamir-PPPVSS-ro}
\end{figure}


\begin{theorem}
  $\Lambda_{RO}^{packed}$ is secure
\end{theorem}
\begin{proof}
  We want to prove the \textit{Correctness, Verifiability} and \textit{IND1-Secrecy} properties and it follows 
  directly as in the PPVSS $\Lambda_{RO}$.
  \begin{itemize}
    \item \textbf{Correctness}: Assume the dealer $D$ and parties $\{P_i\}_{i=1}^n$ follow the protocol. Given a 
      setup from initial phase of the protocol for an input $1^\lambda$ we have $(y_0,y_1,\dots,y_n,(d,z(X)))$ from 
      the sharing phase of the $\Lambda_{RO}^{packed}$ where 
      $y_0=\prod_{j=-(\ell-1)}^{0}h_j^{f_j}$ (or $y_0=\prod_{j=-(\ell-1)}^0PK_j^{f_j}$), $y_i=PK_i^{f_i}$, $d$ is an output 
      from the random oracle $\mathcal{H}$ for the input $(y_0,y_1,\dots,y_n,c_0,c_1,\dots,c_n)$ and $z(X)=r(x)+df(X)$ 
      is a $t+\ell-$degree polynomial such that $f(X)$ is the secret polynomial, $r(X)$ is the polynomial (also secret to $D$) 
      chosen uniformly at random and $c_0=\prod_{j=-(\ell-1)}^{0}h_j^{r(j)}$ (or $y_0=\prod_{j=-(\ell-1)}^0PK_j^{r(j)}$), $c_i=PK_i^{r(i)}$.\par 

      Now with $(y_0,y_1,\dots,y_n,(d,z(X)))$, a verifier checks that the following part of the verification phase evaluates to $d$ 
      for $\ell$ commitment keys:
      \begin{align*}
        &\mathcal{H}(y_0,y_1,\dots,y_n,\prod_{j=-(\ell-1)}^{0}\frac{h_j^{z(j)}}{y_0^d},\frac{PK_1^{z(1)}}{y_1^d},\dots,\frac{PK_n^{z(n)}}{y_n^d})\\
        =&\mathcal{H}(y_0,y_1,\dots,y_n,\prod_{j=-(\ell-1)}^{0}\frac{h_j^{r(j)+df(j)}}{h_j^{df(j)}},\frac{PK_1^{r(1)+df(1)}}{PK_1^{df(1)}},\dots,\frac{PK_n^{r(n)+df(n)}}{h_n^{df(n)}})\\
        =&\mathcal{H}(y_0,y_1,\dots,y_n,\prod_{j=-(\ell-1)}^{0}h_j^{r(j)},PK_1^{r(1)},\dots,PK_n^{r(n)}).
      \end{align*}
      Verifier computes $d$ even when $\ell$ public keys of some target entities are used instead of $\ell$ commitment 
      keys.\par 
      
      Moreover, the reconstruction phase always yields $\{g^{f_j}\}_{j=-(\ell-1)}^0$ in both the approaches. Explicitly, it is 
      clear in the optimistic phase that yields $\{g^{f_j}\}_{j=-(\ell-1)}^0$ after one checks $y_0=\prod_{j=-(\ell-1)}^{0}h_j^{f_j}$ (or $y_0=\prod_{j=-(\ell-1)}^0PK_j^{f_j})$ 
      when given $(g,\{h_j\}_{j=-(\ell-1)}^0,y_0,\{f_j\}_{j=-(\ell-1)}^0)$. The Pessimistic case also yields 
      $\{g^{f_j}\}_{j=-(\ell-1)}^0$ which inherently is the reconstruction step from the PVSS $\Pi_S$ \cite{cryptoeprint:2023/1669}. 
      In essence, we just proved that if the dealer and parties follow the protocol, then verification step returns 
      true and the Reconstruction phase returns the actual secrets.
    \item \textbf{Verifiability}: We need to show that if verify algorithm returns true then for $1\leq i\leq n$, 
      $y_i$ is a valid encryption of $i$'th share of the secrets $\{g^{f_j}\}_{j=-(\ell-1)}^0$ which is intended to party $P_i$, $y_0$ 
      is the commitment of $\ell$ secrets and reconstruct algorithm returns $\{g^{f_j}\}_{j=-(\ell-1)}^0$ irrespective of 
      the two approaches.\par

      Observe the NIZK PoK of PDL \ref{subsec:polynomial-dl}, $\pi_{PDL}=(d,z(X))$ for thane following two cases;
      \begin{itemize}
        \item Suppose $\ell$ public keys of target entities are used to generate $y_0$, then $\pi_{PDL}$ gives a ZKP 
          for a witness of $(g,-(\ell-1),\dots,0,1,\dots,n,F_1,\dots,,SK_1f_1,\dots,SK_2f_2)$
      \end{itemize}
      it gives a ZKP of a witness for 
      
  \end{itemize}
\end{proof}
 
In the next chapter, we will revisit the randomness beacon \cite{cryptoeprint:2020/644} protocol based on PVSS, 
where we replace the PVSS based on Packed Shamir secret sharing with our PPPVSS. 

%%% Local Variables: 
%%% mode: latex
%%% TeX-master: "thesis"
%%% End: 
