% \section{Publicly Verifiable Secret Sharing (PVSS)}
% \label{sec:pvss}
% Publicly Verifiable Secret Sharing (PVSS) is an extension of Non-Interactive Verifiable Secret Sharing 
% (NI-VSS). Unlike NI-VSS where only the parties who possess the secret shares can verify the 
% correctness of the secret sharing, anyone including external entities can verify the correctness of 
% the secret sharing in PVSS.\par 

% Given a $t$ degree polynomial $f\in\mathbb{Z}_q[X]$, let $\mathcal{P}:\mathbb{Z}_q\rightarrow\mathbb{G}$ be a group 
% isomorphism where $0\mapsto g$ with $g$ being a random generator of $\mathbb{G}$ and 
% $Enc:\mathbb{G}\rightarrow\mathbb{G}$ be a public-key encryption scheme and let $\mathcal{P}(f_0)$ be the 
% secret to be secret shared where $f_0=f(0)$. Find a generalization of PVSS in figure \ref{fig:pvss}.

% \input{figures/pvss.tex}

% \begin{theorem}
  
% \end{theorem}


% In practice, there are two types of PVSS schemes, where 
% the secret is the first type if $f_0$ and in the latter type the secret is $g^{f_0}$ for some generator 
% $g\in\mathbb{G}$ and $f_0=f(0)$. In this report, we will recall the latter type of PVSS schemes.

\section{Pre-Constructed Publicly Verifiable Secret Sharing (PPVSS)}
\label{sec:ppvss}
PPVSS was first introduced in \cite{../cryptoeprint:2025/576}, which is used as a building block to 
construct a new e-voting protocol based on Schoenmakers' PVSS \cite{../5581ccd9530540479539d21d1d39ae96}. 
Interestingly, the authors in \cite{../cryptoeprint:2025/576} observed that the original e-voting protocol 
published in 1999 by Schoenmakers is unusually efficient to be just based on a PVSS, which led them to 
discover that Schoenmakers PVSS is actually a PPVSS. What sets PPVSS apart from standard PVSS schemes 
is that it can be used to construct versatile applications, such as e-voting, and can also improve 
efficiency of some existing protocols. The subtle difference between PPVSS and PVSS is that the secret 
itself is committed by the prover along with all its corresponding secret shares.

\section{Conclusion}
The final section of the chapter gives an overview of the important results
of this chapter. This implies that the introductory chapter and the
concluding chapter don't need a conclusion.

%%% Local Variables: 
%%% mode: latex
%%% TeX-master: "thesis"
%%% End: 
