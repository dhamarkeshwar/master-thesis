\chapter{Preliminaries and Literature Review}
\label{chap:preliminaries}

As one of the goals of this chapter is to give enough background to understand the mathematical 
security guarantees from packed Shamir secret sharing, some basic concepts of coding theory 
and Reed-Solomon codes will be discussed in subsequent sections. Basics of sigma protocols 
will be recalled as many Verifiable secret sharing (VSS) schemes security guarantees 
are fundamentally based on them. Lastly, Pre-Constructed Publicly Verifiable Secret Sharing (PPVSS) 
is recalled in this chapter as the final goal of this thesis is to make use of their 
functionalities to build and improve existing applications.

\section{Notation}
Let $(\mathbb{G},\times)$ be a cyclic group of prime order $q$ with hard Discrete Log (DL) and its generator 
being $g$. 
% isomorphic to a subgroup of the multiplicative modular group $\mathbb{Z}_p^*$, where $p$ is prime. 
Also, we write $\mathbb{Z}_{q}[X]_d$ to denote the set of all $d$ degree 
polynomials univariate in $X$ with coefficients in the finite field $\mathbb{Z}_q$. For the remainder of this 
chapter we let $n>t$ for some positive integers $n$ and $t$.

\section{Coding Theory}
\label{sec:linear-codes}
This subsection is a brief recall of linear codes and their properties.

\begin{definition}[Codeword]
  A \textbf{codeword} of length $n$ is a vector $c\in \mathbb{Z}_q^n$.
\end{definition}

\begin{definition}[Linear Code]\cite{gallian2024contemporary}
  If $\mathcal{C}$ be a vector subspace of $\mathbb{Z}_q^n$ with dimension $k$, then $\mathcal{C}$ is 
  said to be a \textbf{linear code}(/ linear $q-$ary code) of length $n$ and dimension $k$.
\end{definition}

In the remainder of the subsection, we let $\mathcal{C}$ be a linear $q-$ary code of length $n$ and 
dimension $k$.

\begin{definition}[Hamming distance]
  The hamming distance $d$ of two codewords of equal length is the number of positions at which the 
  codewords differ. Also, the hamming distance of $\mathcal{C}$, $d(\mathcal{C})$ is defined to be the 
  minimum hamming distance of any two codewords in $\mathcal{C}$.
\end{definition}

\begin{definition}[Hamming weight]
  The hamming weight $wt$ of a codeword $c$ is the number of non-zero positions in $c$. Also, the hamming weight of
  $\mathcal{C}$, $wt(\mathcal{C})$ is defined to be the minimum hamming weight of any codeword in $\mathcal{C}$.
\end{definition}

\begin{lemma}\cite{gallian2024contemporary}\label{lem:distance}
  Given a tuple of codewords of equal length $n$, $(u, v, w)$, let $d(u,v)$ and $wt(w)$ denote the hamming 
  distance of $u, v$ and the hamming weight of $w$ respectively. Then $d(u,v) = wt(u-v)$ and 
  $d(u,v)\leq d(u,w)+d(w,v)$.
\end{lemma}

\begin{definition}[error]
  A vector $r$ is said to be an \textbf{error} of a codeword $c\in\mathcal{C}$ if 
  $r=c+e$ for some $e\neq 0$ and $e$ is called error term of $r$.
\end{definition}

It is trivial to observe that the hamming distance of error $r$ of $c\in\mathcal{C}$ is the minimum of the 
hamming distances of $r$ with each codeword in $\mathcal{C}$.

\begin{definition}[Detectable Error]
  An error $r$ of $c\in\mathcal{C}$ is said to be \textbf{detectable} in $\mathcal{C}$ if 
  $r\notin\mathcal{C}$, otherwise it is said to be an \textbf{undetectable}.
\end{definition}

\begin{theorem}\cite{gallian2024contemporary}\label{th:detectable-error}
  An error $r$ of $c\in\mathcal{C}$ is detectable if the hamming distance of $c$ 
  and $r$ is less than the hamming distance of $\mathcal{C}$, more precisely 
  $d(r,c)<d(\mathcal{C})$.
\end{theorem}
\begin{proof}
  Consider the negation of the statement, i.e., the hamming distance of $c$ and $r$ is less than 
  the hamming distance of $\mathcal{C}$ and $r$ is an undetectable error in $\mathcal{C}$, mathematically 
  we have $d(r,c)<d(\mathcal{C})$ and $r\in\mathcal{C}$. The distance of any two codewords in 
  $\mathcal{C}$ should be at least $d(\mathcal{C})$ implying $d(r,c)\geq d(\mathcal{C})$, which is a 
  contradiction to the negation of our statement.
\end{proof}

The theorem \ref{th:detectable-error} says that any error of a codeword in $\mathcal{C}$ is 
detectable as long as their hamming distance is strictly less than the hamming distance of
$\mathcal{C}$ itself.

\begin{definition}[Correctable Error]
  A detectable error $r$ of $c\in\mathcal{C}$ is said to be \textbf{correctable} if one can obtain its 
  error term $e$ such that $c+e=r$.
\end{definition}

\begin{theorem}\cite{gallian2024contemporary}\label{th:correctable-error}
  One can find the error term $e$ of the detectable error $r$ of $c\in\mathcal{C}$ if 
  $wt(e)<\frac{d(\mathcal{C})}{2}$.
\end{theorem}
\begin{proof}
  We have the following triangular inequality from lemma \ref{lem:distance} for any $w\in\mathcal{C}$ with 
  $w\neq c$:
  \begin{align}\label{eq:triangular-inequality}
    d(\mathcal{C})\leq d(w,c)&\leq d(w,r)+d(r,c).
  \end{align}
  From the equation \eqref{eq:triangular-inequality}, we get 
  \begin{align}\label{eq:triangular-inequality-2}
    d(w,r)\geq d(\mathcal{C})-d(r,c).
  \end{align} 
  Since, $w\neq c$ we always will have 
  \begin{align}\label{eq:triangular-inequality-3}
    d(w,r)>d(r,c).
  \end{align}. 
  From the equations \eqref{eq:triangular-inequality-2} and \eqref{eq:triangular-inequality-3}, we have the 
  following result:
  \begin{align*}
    d(\mathcal{C})-d(r,c)>d(r,c) \implies d(\mathcal{C})>2d(r,c) \iff d(\mathcal{C})>2wt(e).
  \end{align*}
\end{proof}

If one wants to correct a detectable error of a codeword in $\mathcal{C}$ then from theorem 
\ref{th:correctable-error}, its hamming distance with the codeword should be strictly less than half 
the hamming distance of $\mathcal{C}$ itself.

\begin{definition}[Dual Code]
  The vector subspace $\mathcal{C}^{\perp}$ is called a Dual (Code) of $\mathcal{C}$ if it is 
  orthogonal to $\mathcal{C}$.
\end{definition}

\begin{definition}[Generating Matrix]
  The $k\times n-$matrix $\mathcal{G}$ is said to be a generating matrix of $\mathcal{C}$ if it 
  generates $\mathcal{C}$, more precisely, the rows of $G$ form a basis for $\mathcal{C}$. Also, $\mathcal{G}$ is said to be in its 
  \textbf{standard form} if it is of the form
  \begin{align*}
    \mathcal{G} = \begin{bmatrix}
      I_k & P
    \end{bmatrix},
  \end{align*}
  where $I_k$ is the $k\times k$ identity matrix and $P$ is some $k\times (n-k)$ matrix.
\end{definition}

\begin{definition}[Parity Check Matrix]
  Consider the linear transformation $\phi$ as follows:
  \begin{align*}
    \phi:& \mathbb{Z}_q^n \rightarrow \mathbb{Z}_q^{n-k},
  \end{align*}
  where kernel of $\phi$ is $\mathcal{C}$. Then the matrix associated to $\phi$, $\mathcal{H}$, 
  is called the \textbf{parity check matrix} of $\mathcal{C}$.
\end{definition}

\begin{lemma}\cite{gallian2024contemporary}
  If $\mathcal{G}$ is a generating matrix of $\mathcal{C}$ in its standard form, i.e., $\mathcal{G} = \begin{bmatrix}
    I_k & P
  \end{bmatrix}$, then $\mathcal{H}$ being a parity check matrix of $\mathcal{C}$ is given by
  \begin{align*}
    \mathcal{H} = \begin{bmatrix}
      -P^T & I_{n-k}
    \end{bmatrix},
  \end{align*}
  where $I_{n-k}$ is the $(n-k)\times (n-k)$ identity matrix where $P^T$ is the transpose of $P$.
\end{lemma}

One can easily check if a codeword is in $\mathcal{C}$ by multiplying it with its corresponding parity 
check matrix $\mathcal{H}$.

\subsection{Reed Solomon Codes}
\label{subsec:reed-solomon}
Consider the set of all univariate polynomials in $X$ of degree at most $t$ over $\mathbb{Z}_q$, denoted by 
$\mathbb{Z}_q[X]_{\leq t}$. It is trivial to observe that $\mathbb{Z}_q[X]_{\leq t}$ is isomorphic to 
the $t+1$ dimensional vector space $\mathbb{Z}_q^{t+1}$, where each vector consists the coefficients of 
a unique polynomial in $\mathbb{Z}_q[X]_{\leq t}$. Now, consider the following set of codewords 
determined by the evaluation of the polynomials in $\mathbb{Z}_q[X]_{\leq t}$ at $n$ \textbf{distinct} 
points $x_1,\dots,x_n\in \mathbb{Z}_q$:

\begin{align}\label{eq:reed-solomon code}
  RS &= \{(f(x_1),\dots,f(x_n)) : f\in \mathbb{Z}_q[X]_{\leq t}, x_i\neq x_j \text{ for }i\neq j\}.
\end{align}  

\begin{lemma}
  Assume $n>t$. The hamming distance of any two codewords in $RS$ is at least $n-t$. Furthermore, $RS$ is 
  a linear code of length $n$ and dimension $t+1$.
\end{lemma}
\begin{proof}
  For $n>t$, saying that the hamming distance of any two codewords is at least $n-t$ is same as saying 
  that the two codewords in $RS$ are equal in at most $t$ positions. Assume by contradiction that 
  there exists two \textbf{distinct} $t$ degree polynomials $f,g$ in $\mathbb{Z}_q[X]$ with corresponding 
  codewords in $RS$ are equal in $t+1$ positions, i.e., their hamming distance is $n-t-1$. As 
  $\mathbb{Z}_q$ is an integral domain, any $t+1$ distinct points in the set 
  $\mathbb{Z}_q\times\mathbb{Z}_q$ will determine a unique $t$ degree polynomial in $\mathbb{Z}_q[X]_t$. 
  As a consequence, we should have $f=g$ in $\mathbb{Z}_q[X]$ which is a contradiction.\par 

  The remainder of the proof is by a consequence of the first part. More precisely, each codeword in 
  $RS$ determined by a polynomial in $\mathbb{Z}_q[X]_{\leq t}$ is actually a unique representation of the 
  polynomial itself as it consists at least $t+1$ distinct evaluations of that polynomial where $n>t$. 
  That is, $RS$ is isomorphic to $\mathbb{Z}_q[X]_{\leq t}$ as a vector space which has dimension $t+1$.
\end{proof}

\begin{definition}[Reed Solomon Code]
  The $q-$ary linear code $RS$ of length $n$ and dimension $t+1$ defined in \eqref{eq:reed-solomon code} 
  with minimum hamming distance $n-t$ is called a \textbf{$[n, t+1, n-t]-$Reed Solomon Code}
  \cite{doi:10.1137/0108018} in $\mathbb{Z}_q$.
\end{definition}

\begin{corollary}\label{cor:detectable-correctable RS}
  All errors in a $[n, t+1, n-t]-$Reed Solomon (RS) code are detectable if their hamming distances with 
  RS are at most $t$ and $2t<n$. Moreover, all errors of hamming distance with RS being at most $t$ can 
  be corrected if $3t<n$.
\end{corollary}
\begin{proof}
  We have that any error of RS has hamming distance at most $t$ with RS. From theorem 
  \ref{th:detectable-error}, any error of a codeword in $RS$ is detectable if its hamming 
  distance is strictly less than the hamming distance of $RS$ itself, i.e., $t<n-t$, which is equivalent 
  to $2t<n$.\par

  To be able to correct the errors of hamming distance at most $t$ with RS, we need $t<\frac{n-t}{2}$ from 
  theorem \ref{th:correctable-error} which is equivalent to $3t<n$.
\end{proof}

In this report, we will use $[n, t+1, n-t]-$Reed Solomon code of the following form:
\begin{align}\label{eq:reed-solomon code-2}
  RS &= \{(f(1),\dots,f(n)) : f\in \mathbb{Z}_q[X]_{\leq t}\}.
\end{align}

The importance of the Reed-Solomon codes is its direct correspondence with Shamir's 
secret sharing scheme. This allows to trivially analyze their security guarantees. Therefore, the 
packed Shamir secret sharing scheme will now be recalled in the next section \ref{sec:packed-shamir}. 

\section{Packed Shamir Secret Sharing}
\label{sec:packed-shamir}
$(n,t,\ell)$-Packed Shamir secret sharing (\cite{10.1145/129712.129780},\cite{crypto-1984-905})
 scheme is a threshold secret sharing scheme which is a variant of $(n,t)$-Shamir's 
 secret sharing scheme \cite{10.1145/359168.359176} where $n>2t+\ell-1$. In a nutshell, the $t+\ell-1$ 
 degree secret polynomial with coefficients in $\mathbb{Z}_q$ which evaluates to $\ell$ 
 secrets is secret shared amongst $n$ parties such that any $t+\ell$ parties can 
 reconstruct back the secret polynomial. Recall that Shamir's secret sharing scheme 
 requires at least $t+1$ parties to reconstruct the secret polynomial in contrast to 
 the $t+\ell$ parties in the Packed Shamir secret sharing scheme. The scheme is
 summarized in the Figure \ref{fig:packed-shamir}.
\begin{figure}[ht]
    \centering
    \begin{tcolorbox}[title=\textbf{Packed Shamir Secret Sharing}, width=0.9\textwidth, colframe=blue!75!black, colback=blue!10, sharp corners]
        Given $\ell$ secrets to share amongst $n$ parties, where at most $t$ of them can
        be \textit{(passively)} corrupt, the $(n,t,\ell)$-Packed Shamir secret sharing scheme description is as
        follows:
        
        \vspace{0.5em}
        \textbf{Sharing Algorithm:}
        \begin{itemize}
            \item Dealer constructs the secret polynomial $f\in\mathbb{Z}_q[X]_{t+l-1}$
                  via the lagrange interpolation by choosing $t+\ell$ elements in 
                  $\mathbb{Z}_q$ where $\ell$ of them are secrets, $\{s_i\}_{i=0}^{\ell-1}$, with
                  $f(-i)=s_i$ for all $i$ and remaining $t$ are chosen uniformly 
                  at random in $\mathbb{Z}_q$.
            \item Each party $P_i$ receives their share $f(i)$ from the Dealer
                  for each $i\in\{1,\dots,n\}$
        \end{itemize}
        
        \vspace{0.5em}
        \textbf{Reconstruction Algorithm:}
        \begin{itemize}
            \item Any $\mathcal{Q}$ set containing at least $t+\ell$ parties can use the 
            lagrange interpolation to compute $\{s_i\}_{i=0}^{\ell-1}$ as follows:
            \begin{align*}
                s_m &= \sum_{i\in \mathcal{Q}} f(i) \left[\prod_{j\in \mathcal{Q}, j\neq i}\frac{-m-j}{i-j}\right] &&, m\in\{0,\dots,\ell-1\} \\
            \end{align*}
            \item The secrets $\{s_i\}_{i=0}^{\ell-1}$ are outputted as the result.
        \end{itemize}
    \end{tcolorbox}
    \caption{Packed Shamir Secret Sharing}
    \label{fig:packed-shamir}
\end{figure}

One can observe that all the secret shares of a secret polynomial chosen by the dealer form a codeword in 
$[n,t+\ell,n-t-\ell+1]-$RS code. If the adversary is malicious and can corrupt at most $t$ parties, then from 
corollary \ref{cor:detectable-correctable RS}, the honest shareholders can detect the errors if 
$2t\leq n-\ell$ and moreover all such errors can be corrected if $3t\leq n-\ell$. Also, one can use 
the Berlekamp-Welch algorithm \cite{welch1986error} to correct the errors.\par

But Shamir secret sharing scheme is designed particularly to defend against passive adversaries and not 
against malicious adversaries. A class of threshold secret sharing schemes which are designed to defend 
against malicious adversaries is Verifiable Secret Sharing (VSS). There are many VSS schemes in the literature, 
and as of writing this report the efficient VSS schemes based on Shamir secret sharing are $\Pi_F$, 
$\Pi_P$ and $\Pi_{LA}$ \cite{cryptoeprint:2023/1669} which are alternatives to the original VSS schemes 
from Feldman \cite{DBLP:conf/focs/Feldman87}, Pedersen \cite{crypto-1991-1671} and the more recent 
ABCP \cite{cryptoeprint:2023/992}. VSS schemes based on Shamir secret sharing allow shareholders to 
verify the correctness of the shares obtained during both the sharing and reconstruction phases. This enables 
these VSS schemes to defend against malicious adversaries who can corrupt $t$ parties as long as 
$t\leq\frac{n-1}{2}$ (In contrast to $t<\frac{n}{3}$ in Shamir secret sharing), and as a 
result it can detect the exact entity who is deviating from the protocol. Publicly Verifiable Secret Sharing (PVSS) is an extension of VSS where anyone can 
verify the validity of the secret shares during the sharing phase. More recently, Pre-Constructed Publicly 
Verifiable Secret Sharing (PPVSS)\cite{cryptoeprint:2025/576} was proposed which is an extension of PVSS. 
The main tools used in VSS, PVSS and PPVSS schemes are the \textbf{Zero Knowledge Proofs} which we overview in the 
next section \ref{sec:sigma-protocols}.

\section{Zero Knowledge Proofs}
\label{sec:sigma-protocols}
The agenda of this subsection is to give a brief formal background about some important primitives 
used in VSS $\Pi_{F},\Pi_{P},\Pi_{LA}$, the PVSS $\Pi_S$ \cite{cryptoeprint:2023/1669}, and the PPVSS $\Lambda_{RO}$ \cite{cryptoeprint:2025/576}, schemes. 
A \textit{Zero Knowledge Argument of Knowledge} (ZK AoK) is a protocol between a prover and a verifier
where the prover tries to convince the verifier that a statement is true without revealing any information 
about why it is true. Unlike the Zero Knowledge Proof of Knowledge (ZK PoK) where the prover cannot cheat 
the verifier even with unbounded computational power, a ZK AoK requires the prover to be computationally bounded 
to not be able to cheat any verifier. More about ZK AoK and ZK PoK can be found in \cite{cryptoeprint:2017/1066}.\par

Let $Y,X$ and $W$ be three sets with $R$ being a ternary relation on $Y\times X\times W$. Consider the 
three PPT interactive algorithms (\textit{Setup}, $P$, $V$), where \textit{Setup} returns a 
common reference string (CRS) $\sigma$ when input $1^{\lambda}$. Given $\sigma$, the following 
is the CRS-dependent language of the relation $R$.
\begin{align*}
  L_{\sigma}=\{x\in X :\exists w\in W, (\sigma, x, w)\in R\},
\end{align*}
where $w$ is called a witness for a statement $x$ if $(\sigma, x, w)\in R$. Also, let $\mathcal{R}$ be 
a PPT algorithm that returns an element in the relation $R$ when input $1^{\lambda}$.\par

Given a relation $R$ and its corresponding language, a \textbf{$\sum$-protocol} 
is a $3$-round \textit{interactive} protocol between two Probabilistic Polynomial Time (PPT) algorithms, 
a prover $P$ 
and a verifier $V$. For some statement in the language of $R$, in the first round $P$ sends a 
commitment $a$ to $V$. To which $V$ sends a challenge $d$ to $P$ in the second round 
and finally $P$ responds back with the response $z$ to $V$ in the third round. 
$V$ outputs \textbf{true} or \textbf{false} upon the proof verification on transcript
$trans := (a, d, z)$. 
Informally, with a $\sum-$protocol a prover $P$ tries to convince
a verifier $V$ that they know a witness $w$ for a given statement $x$ in the language without 
revealing any information about $w$. To state it formally, a $\sum-$protocol is 
supposed to satisfy \textit{completeness}, \textit{Honest Verifier Zero Knowledge} (HVZK)
and \textit{Special Soundness} which are defined as follows.

\begin{definition}[Completeness]
  A $\sum-$protocol is said to be \textbf{complete} for $\mathcal{R}$ if
  the honest verifier $V$ always accepts the honest prover $P$ for any statement in the language 
  defined by $R$.
\end{definition}

\begin{definition}[HVZK]
  A $\sum-$protocol is said to be \textit{Honest Verifier Zero Knowledge}(\textbf{HVZK}) for $\mathcal{R}$ if there exist a PPT algorithm $\mathcal{S}$ 
  that simulates $trans$ of the scheme corresponding to a given statement, $x$, in the 
  language that has witness $w$. 
  That is, given $x$,
  \begin{align*}
    trans(P(x,w)\leftrightarrow V(x)) &\approx trans(\mathcal{S}(x) \leftrightarrow V(x))&& \text{, for any witness $w$ of $x$.}
  \end{align*}
  Where $trans(P(\cdot)\leftrightarrow V(\cdot))$ is the transcript of the $\sum-$protocol amongst 
  $P$ and $V$ and $\approx$ denotes the indistinguishability of the two transcripts.
\end{definition}

\begin{definition}[Special Soundness]
  A $\sum-$protocol is said to satisfy \textbf{Special Soundness} for $\mathcal{R}$, 
  if there exists a PPT extractor $\mathcal{E}$ for any two valid transcripts, $(a,d,z)$ and 
  $(a,d',z')$, corresponding to a given statement $x$ in the language with only a unique 
  witness $w$ and $d\neq d'$ such that $\mathcal{E}(a,d,z,d',z')$ outputs the witness $w$. 
\end{definition}

It is shown that a public-coin, complete, HVZK, special soundness $\sum-$protocol can be made into a
Non Interactive Zero Knowledge (NIZK) Proof of Knowledge (PoK) or Argument of Knowledge (AoK) in the 
Random Oracle ($RO$) model using Fiat-Shamir transform \cite{10.1007/3-540-47721-7_12}. 
In the following subsections, two important NIZK PoK schemes are recalled which were used in $\Pi_S$ and 
$\Lambda_{RO}$ schemes and we are going to use them to build our protocols introduced in the next 
chapter. 
% Also, we will introduce a NIZK AoK scheme which will be used in one of our new protocols.

\subsection{Chaum-Pedersen Protocol for DL Equality}
\label{subsec:chaum-pedersen}
Recall $\mathbb{G}$ being the cyclic group of prime order $q$ with hard Discrete Logarithm (DL). 
For some $g,h\in \mathbb{G}$ consider the following relation:
\begin{align*}
  R_{DLEQ} &= \{(g,h,a,b),x : a=g^x, b=h^x\}.
\end{align*}
In \cite{10.1007/3-540-48071-4_7}, Chaum and Pedersen proposed a NIZK PoK scheme for the DL Equality 
relation, $R_{DLEQ}$. Informally, a prover $P$ can convince a verifier $V$ that they know $x$ such that
it can be used with both $g$ and $h$ to obtain $a$ and $b$ respectively. This protocol is widely used in
many cryptographic applications like threshold decryption, e-voting and Randomness Beacons. 
We summarize the protocol in Figure \ref{fig:chaum-pedersen}.
\begin{figure}[ht]
    \centering
    \begin{tcolorbox}[title=\textbf{Chaum-Pedersen Protocol for DLEQ}, width=0.9\textwidth, colframe=blue!75!black, colback=blue!10, sharp corners]
        % Let $(g,h,a,b)\in L_{DLEQ}$ be a statement with its corresponding witness being $x$ where $L_{DLEQ}$
        % is the language defined by the relation $R_{DLEQ}$ and 
        Let $g$ be the group generator sampled in the setup phase, $[(g,h,a,b),x]\in R_{DLEQ}$ and 
        $\mathcal{H}$ be a Random Oracle ($RO$).\par
        \vspace{0.5em}
        \textbf{Prover}
        \begin{itemize}
            \item Samples $r\in_{R}\mathbb{Z}_q$ uniformly at random and sets 
                $c_1=g^r$ and $c_2=h^r$.
            \item Sets $d\leftarrow \mathcal{H}(a,b,c_1,c_2)$, where $\mathcal{H}$ is 
                an agreed upon Random Oracle ($RO$).
            \item Sets $z\equiv r+dx \pmod{q}$ and returns the proof(/transcript) $\pi:= (d,z)$.
        \end{itemize}
        
        \vspace{0.5em}
        \textbf{Verifier}
        \begin{itemize}
            \item Checks if $d\leftarrow \mathcal{H}(a,b,\frac{g^z}{a^d},\frac{h^z}{b^d})$ 
                and outputs \textbf{true} or \textbf{false} accordingly.
        \end{itemize}
    \end{tcolorbox}
    \caption{Chaum-Pedersen NIZK PoK for DLEQ}
    \label{fig:chaum-pedersen}
\end{figure}


\subsection{NIZK PoK for Polynomial DL}
\label{subsec:polynomial-dl}
Recall $\mathbb{G}$ being the cyclic group of prime order $q$ with hard Discrete Logarithm (DL) and $g$ 
being its generator. Consider the following relation for some polynomial $f\in\mathbb{Z}_q[X]_t$ with
degree $t<n$:
\begin{align*}
  R_{PDL} &= \{(g,x_1,\dots,x_n,F_1,\dots,F_n),f(X) : (F_i=g^{f(x_i)}, 1\leq i\leq n)\wedge(x_i\neq x_j, i\neq j)\}.
\end{align*}
The $R_{PDL}$ is based on the Polynomial Discrete Logarithm (PDL) formally introduced in \cite{cryptoeprint:2023/1669}, 
and its definition is recalled in the following.

\begin{definition}[Polynomial Discrete Logarithm]
  Let $g$ be a generator for the prime order $q$ cyclic group $\mathbb{G}$. Given $F_1,\dots,F_n$ and distinct elements $x_1,\dots,x_n$ in 
  $\mathbb{Z}_q$, find a polynomial $f\in\mathbb{Z}_q[X]$ with degree at most $t$, where $F_i=g^{f(x_i)}$ 
  for $1\leq i\leq n$ and $t\leq n-1$.\par

  In other words, an algorithm $\mathcal{A}$ is said to have an advantage $\epsilon$ in solving PDL if 
  \begin{align*}
    Pr[\mathcal{A}(x_1,\dots,x_n,g,g^{f(x_1)},\dots,g^{f(x_n)})]\geq\epsilon,
  \end{align*}
  where $f\in\mathbb{Z}_q[X]$ is at most a $t$ degree polynomial with $t\leq n-1$ and the probability is over a 
  chosen random generator $g$ of $\mathbb{G}$ with $q=|\mathbb{G}|$ being prime and distinct $x_1,\dots,x_n$ 
  elements in $\mathbb{Z}_q$.
\end{definition}

Based on the hardness of PDL, a NIZK PoK scheme $\pi_{PDL}$ was proposed in \cite{cryptoeprint:2023/1669} which is 
summarized in figure \ref{fig:polynomial-dl}.

\begin{figure}[ht]
    \centering
    \begin{tcolorbox}[title=\textbf{A NIZK PoK for Polynomial DL}, width=0.9\textwidth, colframe=blue!75!black, colback=blue!10, sharp corners]
        Let $(g,x_1,\dots,x_n,F(x_1),F(x_n))\in L_{PDL}$ be a statement with its corresponding witness being $f\in\mathbb{Z}_q[X]_t$ 
        where $L_{PDL}$ is the language defined by the relation $R_{PDL}$.
        
        \vspace{0.5em}
        \textbf{Prover}
        \begin{itemize}
            \item Samples $r\in_{R}\mathbb{Z}_q[X]_t$ uniformly at random and sets 
                $\Gamma_i=g^{r(x_i)}$ for $1\leq i\leq n$.
            \item Sets $d\leftarrow \mathcal{H}(F_1, \dots, F_n, \Gamma_1, \dots, \Gamma_n)$, where $\mathcal{H}$ is 
                an agreed upon Random Oracle (RO).
            \item Sets $z(X)\equiv r(X)+df(X) \pmod{q}$ and returns the proof(/transcript) $\pi:= (d,z(X))$.
        \end{itemize}
        
        \vspace{0.5em}
        \textbf{Verifier}
        \begin{itemize}
            \item First checks if $z$ is a $t$ degree polynomial in $\mathbb{Z}_q[X]_t$. If so, they proceed with the next step.
            \item Checks if $d\leftarrow \mathcal{H}(F_1, \dots, F_n,\frac{g^{z(x_1)}}{F_1^d}, \dots, \frac{g^{z(n)}}{F_n^d})$. 
            \item If first two steps are correct then they output \textbf{true}, otherwise \textbf{false}.
        \end{itemize}
    \end{tcolorbox}
    \caption{A NIZK PoK for Polynomial DL based on Schoenmakers' PVSS}
    \label{fig:polynomial-dl}
\end{figure}


\begin{theorem}\cite{cryptoeprint:2023/1669}\label{th:PDL security}
  Let $[(g,x_1,\dots,x_n,F_1,\dots,F_n),f(X)]\in R_{PDL}$ where $f\in\mathbb{Z}_q[X]_{\leq t}$. 
  Assuming PDL is hard, for $0\leq t\leq n-1$, the protocol $\pi_{PDL}$ \ref{fig:polynomial-dl} (described in figure \ref{fig:polynomial-dl}) 
  is a NIZK PoK for $R_{PDL}$ in the $RO$ model.
\end{theorem}

In \cite{cryptoeprint:2023/1669}, Baghery formally introduced a NIZK PoK scheme for the Polynomial DL 
relation, $R_{PDL}$, which is a generalization of Schnorr's ID protocol \cite{crypto-1989-1727}. Informally, 
a prover $P$ can convince a verifier $V$ that they know a $t-$degree polynomial $f$ such that 
it can be used with $g$ to obtain $F_i$ for $1\leq i\leq n$. A fairly recent application based on 
this protocol is used to construct an efficient e-voting protocol \cite{cryptoeprint:2025/576} where 
the authors have introduced the Pre-Constructed Publicly Verifiable Secret Sharing scheme and used it as a 
building block.

\section{Pre-Constructed Publicly Verifiable Secret Sharing}
\label{sec:ppvss}
Pre-Constructed Publicly Verifiable Secret Sharing (PPVSS) was first introduced in 
\cite{cryptoeprint:2025/576}, which was used as a building block to 
construct an efficient e-voting protocol alternative to Schoenmakers' e-voting protocol. 
Interestingly, the authors of \cite{cryptoeprint:2025/576} observed that the Schoenmakers' e-voting protocol, though 
not efficient in practice, published in 1999 is an unusual application as it is based on a 
PVSS, which led them to discover that the underlying PVSS used is actually a PPVSS. 
What sets PPVSS apart from standard PVSS schemes is that it can be used to build versatile 
applications, such as e-voting, randomness beacons etc., and can 
also improve efficiency of some existing protocols. The subtle difference between PPVSS and 
PVSS is that the secret itself is committed by the prover along with all its corresponding 
secret shares. The relevance of this section is to generalize the PPVSS to the case 
where there is more than one secret and use them to revisit a real time application 
to improve its efficiency in the subsequent chapters. Because of its importance in this thesis, the 
definitions of a PPVSS and an example is recalled in the upcoming sections.\par

\subsection{Definitions}
\label{sec:ppvss-definitions}

The following definitions are directly taken from \cite{cryptoeprint:2025/576}.

\begin{definition}[PPVSS]
    A PPVSS scheme should have four algorithms, namely, Initial, Share, Verify and Reconstruction whose 
    descriptions are as follows:
    \begin{itemize}
        \item \textbf{Initial} $(1^\lambda)\rightarrow(\{PK_i,SK_i\}_{i=1}^n\sqcup\{h_0\text{ or }(PK_0,SK_0)\})$: 
          When given $1^\lambda$, each party $P_i$ for $1\leq i\leq n$ registers their public key $PK_i$ in a 
          public ledger and withholds the corresponding secret key $SK_i$. Also, all parties and the dealer $D$ 
          agree on a commitment key or public key whose secret key is known to a target person. 
          Note that the message space of the public-key scheme is a subgroup of $(\mathbb{G},\times)$.
        \item \textbf{Share} $(n,t,f_0,\{PK_i\}_{i=1}^n\bigsqcup\{h_0\text{ or }PK_0\})$\par
          $\rightarrow(\{y_i\}_{i=0}^n,\pi_{PPVSS})$:\par 
          It secret shares $f_0$ to obtain the shares $\{f_i\}_{i=1}^n$. For $1\leq i\leq n$, uses 
          the public key $PK_i$ to encrypt $f_i$ and obtain the encrypted share $y_i$. Now, it uses the commitment 
          key $h_0$ ( or public key $PK_0$) to commit(/encrypt) $f_0$ to obtain $y_0$. 
          In the next step, it uses a NIZK proof $\pi_{PPVSS}$ protocol to prove that $y_0$ is a valid 
          commitment(/encryption) of the secret and $\{y_i\}_{i=1}^n$ has valid encryptions of the corresponding shares. Finally, it returns $(\{y_i\}_{i=0}^n,\pi_{PPVSS})$.\par
        \item \textbf{Verify} $(n,t,\{y_i\}_{i=0}^n,\pi_{PPVSS})\rightarrow$ \textbf{true/false}: 
          This algorithm(which can be performed by anyone) checks if the NIZK proof $\pi_{PPVSS}$ is valid for 
          $\{y_0,\dots,y_n\}$ and then returns true, otherwise false.\par
        \item \textbf{Reconstruct}: There are two approaches based on cooperation of the dealer $D$ and are as follows:
          \begin{itemize}
            \item \textbf{(Optimistic)} $Reconstruction^{opt}[\{h_0,f_0,r_0,y_0\}\text{ or }\{PK_0,SK_0,y_0\}]$\par$\rightarrow(f_0\text{ or }false)$:
              
              Given the input a verifier checks if $y_0$ is a valid commitment(/encryption) of the secret. 
                 If so, it returns $f_0$; otherwise it returns $false$.
            \item \textbf{(Pessimistic)} $Reconstruction^{pes}[\{y_i,SK_i\}_{i\in\mathcal{Q},|\mathcal{Q}|=t+1}]\rightarrow(f_0\text{ or }false)$: 
              Given any $t+1$ encrypted shares along with corresponding secret keys, it outputs the secret 
              $f_0$ or $false$. This can be done in two phases as follows: 
                \begin{itemize}
                  \item \textbf{Decryption of the shares}, each party $P_i\in\mathcal{Q}$ decrypts $y_i$ to obtain 
                    $f_i$ using its secret key $SK_i$. Then it generates a NIZK proof $\pi_i^{Dec}$ which proves that 
                    $f_i$ is the correct decryption of $y_i$. Now, $P_i$ publishes $(f_i,\pi_i^{Dec})$.
                  \item \textbf{Share pooling}, a verifier $V$ (not necessarily from the shareholders) checks if proof 
                   $\pi_i^{Dec}$ is correct for each $P_i\in\mathcal{Q}$. If any check fails, then $V$ returns $false$; 
                   otherwise $V$ applies a reconstruction procedure to the set of valid shares, $\{f_i\}_{i\in\mathcal{Q},|\mathcal{Q}|=t+1}$, 
                   and returns $f_0$.
                \end{itemize}
          \end{itemize}
    \end{itemize}
\end{definition}

A PPVSS is said to be secure:

\begin{itemize}
  \item \textbf{Correctness}: If the dealer and parties follow the protocol, then the \textit{\textbf{Verify}} 
    algorithm returns \textit{\textbf{true}} and the \textit{\textbf{Reconstruct}} algorithm returns $f_0$ 
    irrespective of which approach. For any integer $n\geq t+1$ with $t\geq0$, a PPVSS is said 
    to be correct for
    $\left(\{PK_i,SK_i\}_{i=1}^n\sqcup\{h_0\text{ or }(PK_0,SK_0)\}\right)\leftarrow$\textbf{Initial} $(1^\lambda)$ when 
    it satisfies the following based on the output of the \textit{Share} algorithm.\par
   When \textit{Share} algorithm outputs a commitment(/encryption) of the secret, then
        \begin{align*}
          Pr\begin{bmatrix}
            (\{y_i\}_{i=0}^n,\pi_{PPVSS})&\leftarrow \textit{\textbf{Share}} \big(n,t,f_0,\{PK_i\}_{i=1}^n\bigsqcup\\
            &\{h_0\text{ or }PK_0\}:\\
            \textit{\textbf{true}}&\leftarrow \textit{\textbf{Verify}} (n,t,\{y_i\}_{i=0}^n,\pi_{PPVSS})
          \end{bmatrix} = 1,
        \end{align*}
    
        \begin{align*}
          Pr\begin{bmatrix}
            (\{y_i\}_{i=0}^n,\pi_{PPVSS})&\leftarrow \textit{\textbf{Share}} \big(n,t,f_0,\{PK_i\}_{i=1}^n\bigsqcup\\
            &\{h_0\text{ or }PK_0\}),\\
            f_0^{'}&\leftarrow Reconstruction^{opt}[\{h_0,f_0,r_0,y_0\}\\
            &\textit{ or }\{PK_0,SK_0,y_0\}\bigvee\\
            f_0^{'}&\leftarrow Reconstruction^{pes}[\{y_i,SK_i\}_{i\in\mathcal{Q},|\mathcal{Q}|=t+1}]:\\
            &f_0^{'}=f_0
          \end{bmatrix} = 1.
        \end{align*}
  \item \textbf{Verifiability}: If \textit{\textbf{Verify}} returns \textit{\textbf{true}}, then 
    \textit{\textbf{(Optimistic)}} $Reconstruct^{opt}$ and/or \textit{\textbf{(Pessimistic)}} $Reconstruct^{pes}$ 
    output being $f_0$ is the actual secret of whom the shares are encrypted. Moreover, 
    $\{y_i\}_{i=1}^n$ are valid encryptions of the shares of same secret with 
    high probability if the following statement is true.\par
    $y_0$ is a valid commitment(/encryption) of the secret with high 
        probability. More formally, given $\lambda$, for any integers 
        $n\geq 2t+1$ and $t\geq0$, a PPVSS is said to be verifiable if for any 
        $\mathcal{PPT}$ adversary $\mathcal{A}$, we have:
        \begin{align*}
          Pr\begin{bmatrix}
            \big(\{PK_i,SK_i\}_{i=1}^n\sqcup&\{h_0\text{ or }(PK_0,SK_0)\}\big)\leftarrow\textbf{ Initial } (1^\lambda),\\
            (\{y_i\}_{i=0}^n,\pi_{PPVSS})&\leftarrow \mathcal{A} \big(n,t,f_0,\{PK_i\}_{i=1}^n\bigsqcup\\
            &\{h_0\text{ or }PK_0\}\big),\\
            f_0^{'}&\leftarrow Reconstruction^{opt}[\{h_0,f_0,r_0,y_0\}\\
            &\textit{ or }\{PK_0,SK_0,y_0\}]\bigvee\\
            f_0^{'}&\leftarrow Reconstruction^{pes}[\{y_i,SK_i\}_{i\in\mathcal{Q},|\mathcal{Q}|=t+1}]:\\
            \textit{\textbf{true}}&\leftarrow \textit{\textbf{Verify}} (n,t,\{y_i\}_{i=0}^n,\pi_{PPVSS})
            \bigwedge f_0^{'}\neq f_0
          \end{bmatrix} \leq negl(\lambda),
        \end{align*}
        where $\mathcal{Q}$ is the set of honest parties.
  \item \textbf{IND1-Secrecy (Indistinguishability of Secrets)}: Before reconstruction phase, any amount of public 
    information along with secret keys of at most $t$ parties excluding $SK_0$ will give 
    absolutely no information about the secret $f_0$. More formally, for integers $n>1$ and 
    $t+1\leq n$, the PPVSS is said to satisfy \textit{IND1-Secrecy} if for any $\mathcal{PPT}$ adversary $\mathcal{A}$ 
    corrupting at most $t$ parties, excluding the owners of $SK_0$, has negligible 
    advantage in the following game played against a challenger.
    \begin{itemize}
      \item The challenger runs \textit{\textbf{Initial}} $(1^\lambda)$ of PPVSS to obtain 
        $\{PK_i,SK_i\}_{i=1}^n\sqcup\{h_0\text{ or }(PK_0,SK_0)\}$ and sends all public 
        information along with secret information of all corrupted parties to $\mathcal{A}$.
      \item The challenger chooses two secrets, $s_0=f_0$ and 
        $s_1=f_0^{'}$ at random in the space of secrets. Furthermore, it chooses $b\in\{0,1\}$ 
        uniformly at random and runs 
        \textit{\textbf{Share}} $(n,t,s_0,\{PK_i\}_{i=1}^n\bigsqcup\{h_0\text{ or }PK_0\})$ 
        algorithm of the PPVSS scheme and obtains $(\{y_i\}_{i=0}^n,\pi_{PPVSS})$. Finally, it 
        sends all public information generated in \textit{Share} phase together with $s_b$.
      \item $\mathcal{A}$ guesses a bit $b'\in\{0,1\}$.
    \end{itemize}
    The advantage of $\mathcal{A}$ is defined to be $|Pr[b'=b]-\frac{1}{2}|$.
\end{itemize}

$\Lambda_{RO}$ is the PPVSS scheme introduced in \cite{cryptoeprint:2025/576}, recalled in 
figure \ref{fig:ppvss}, which is actually based on $\Pi_S$ \cite{cryptoeprint:2023/1669}.\par 
\begin{figure}[ht]
    \centering
    \resizebox{\textwidth}{!}{ % Further reduced the scaling factor to fit the content
    \begin{tcolorbox}[title=\textbf{$\Lambda_{RO}$ \cite{cryptoeprint:2025/576}}, width=1.2\textwidth, colframe=blue!75!black, colback=blue!10, sharp corners]
        
        \textbf{Initialization:}
            All parties $\{P_i\}_{i=1}^n$ and dealer $D$ agree on the prime field $\mathbb{Z}_q$, a group $(\mathbb{G},\times)$ of 
            order $q$ with a generator $g$, random oracle $\mathcal{H}$. Also, each party 
            $P_i$ registers their public key $PK_i$ in the public ledger and all agree on a commitment 
            key or a public key, $PK_0\neq g$, whose secret key is known to a target person.

        \vspace{0.5em}
        \textbf{Share:}
        \begin{itemize}
            \item Dealer $D$ samples a $t$-degree polynomial $f\in\mathbb{Z}_q[X]$ uniformly at random and 
              sets $g^{f_0}$ as the secret to be shared, where $f_0=f(0)$.
            \item For each $1\leq i\leq n$, $D$ computes $f(i)=f_i$ and uses $PK_i$ to encrypt the secret share  
               to obtain $PK_i^{f_i}=y_i$. $D$ also encrypts(/commits) to the secret $g^{f_0}$ using the encryption(/commitment) 
              key $PK_0$ to obtain $y_0=PK_0^{f_0}$. 
            \item $D$ uses the PDL proof scheme \ref{subsec:polynomial-dl} to generate $\pi_{PDL}$ as follows:
                 \begin{itemize}
                    \item Samples a $t-$degree polynomial $r\in\mathbb{Z}_q[X]$ uniformly at random and 
                    computes $c_i=PK_i^{r_i}$ where $r_i=r(i)$ for $0\leq i\leq n$.
                    \item Using $\mathcal{H}$, $d=\mathcal{H}(y_0,\dots,y_n,c_0,\dots,c_n)$ is computed.
                    \item Sets $z(X)=r(X)+df(X)$, hence $\pi_{PDL}=(d,z(X))$ is obtained. 
                 \end{itemize}
            \item $D$ broadcasts the encryptions of the shares along with the encryption(/commitment) of the secret 
              with $\pi_{PDL}$ which proves the validity of the encrypted shares and encrypted(/committed) 
              secret, i.e., broadcasts $\{y_i\}_{i=0}^n$ and $(d,z(X))$.
        \end{itemize}
        
        \vspace{0.5em}
        \textbf{Verification:}
            Given public keys $\{PK_i\}_{i=1}^n$ and public(commitment) key $PK_0$, any entity can check 
            $\pi_{PDL}$ to verify the correctness of the encrypted shares $\{y_i\}_{i=1}^n$ and $y_0$ being the encryption(/commitment) of the secret. 
            They will output \textbf{true} or \textbf{false} based on the verification of the proof. The 
            procedure is outlined as follows:
        \begin{itemize}
            \item The entity checks if $z(X)$ is a $t-$degree polynomial or not.
            \item Checks if $d=\mathcal{H}(y_0,y_1,\dots,y_n,\frac{PK_0^{z(0)}}{y_0^d},\frac{PK_1^{z(1)}}{y_1^d},\dots,\frac{PK_n^{z(n)}}{y_n^d})$.
            \item Outputs \textbf{true} if both of the above checks are satisfied, otherwise \textbf{false}.
        \end{itemize}

        \vspace{0.5em}
        \textbf{Reconstruction:}
            There are two approaches to reconstruct the secret $g^{f_0}$ based on the 
            cooperation of the dealer $D$ which are as follows:
            \begin{itemize}
                \item \textbf{Optimistic Reconstruction:} $D$ publishes $f_0$, then any verifier (not necessarily shareholders) 
                when given $g,PK_0,y_0$ can check if $y_0=PK_0^{f_0}$ and returns $g^{f_0}$ 
                if the check passes, if not they return \textbf{false}.
                \item \textbf{Pessimistic Reconstruction:} If $D$ refuses to reveal $f_0$, then any set 
                $\mathcal{Q}$ consisting at least $t+1$ shareholders will do the following (which 
                is same as the reconstruction step in the PVSS $\Pi_S$ protocol):
                \begin{itemize}
                    \item Each party $P_i\in\mathcal{Q}$ decrypts their share $y_i$ using their private key $SK_i$ 
                      corresponding to $PK_i$ to obtain $g^{f_i}$. Then they publish $g^{f_i}$ 
                      along with a DLEQ proof \ref{subsec:chaum-pedersen}, $\pi_{DLEQ}$ which proves that the 
                       $g^{f_i}$ is the correct decryption of $y_i$.
                    \item If $\mathcal{Q}$ consists at least $t+1$ honest parties, they can use the 
                    lagrange interpolation to compute the secret $g^{f_0}$ as follows:
                    \begin{align*}
                        g^{f_0} &= \prod_{i=1}^{t+1} g^{f_i^{\lambda_i}} = g^{\sum_{i=1}^{t+1}f_i\cdot\lambda_i},\\
                    \end{align*}
                    where $\lambda_i=\prod_{j\neq i}\frac{j}{j-i}$ are lagrange coefficients.
                \end{itemize}
            \end{itemize}
    \end{tcolorbox}
    }
    \caption[PPVSS Scheme]{a Pre-Constructed Publicly Verifiable Secret Sharing (PPVSS) scheme}
    \label{fig:ppvss}
\end{figure}
The security guarantees of $\Lambda_{RO}$ are explained in \cite{cryptoeprint:2025/576} which follows from 
the security guarantees of $\Pi_S$, also the authors used $\Lambda_{RO}$ to build the new e-voting 
protocol alternative to the Schoenmakers' e-voting protocol \cite{5581ccd9530540479539d21d1d39ae96} 
and showed that they achieved improvement in the verification phase 
(7 to 30 times faster in implementation for some parameters). 

\section{Conclusion}
In summary, some basic concepts of linear codes and Reed-Solomon codes were recalled. Next, Packed Shamir secret sharing scheme which is a generalization of the Shamir Secret Sharing scheme \cite{10.1145/359168.359176} 
was recalled and its correspondence with Reed-Solomon codes is discussed to explain how 
its mathematical security guarantees are achieved against passive adversaries. As one 
of the goals of this thesis is to explore threshold secret sharing schemes that commits to 
secret and also 
can defend against malicious adversaries, PPVSS \cite{cryptoeprint:2025/576} was recalled. 
Moving forward, in the next chapter we will introduce 
Packed Pre-Constructed Publicly Verifiable Secret Sharing (PPPVSS) which is a generalization of 
PPVSS and give two examples whose inspiration is drawn from $\Lambda_{RO}$.

%%% Local Variables: 
%%% mode: latex
%%% TeX-master: "thesis"
%%% End: 
