\chapter{Literature Review}
\label{cha:1}
In this chapter we sequently recall Packed Shamir secret sharing, 
Sigma ($\sum$) Protocols and Publicly Verifiable Secret Sharing (PVSS) followed by
the recent scheme introduced in \cite{cryptoeprint:2025/576}, namely, 
Pre-Constructed Publicly Verifiable Secret Sharing (PPVSS) which has versatile 
applications and also improves efficiency in existing applications.
The agenda of this chapter is to give enough background before describing our Packed PPVSS (PPPVSS) scheme 
and its corresponding security guarantees in the next chapter.

\section{Introduction}

\section{Preliminaries}
\label{sec:preliminaries}
\subsection{Notation}
Let $\mathbb{G}$ be a cyclic subgroup of prime order $q$ with its generator being 
$g$, isomorphic to a subgroup of the multiplicative modular group $\mathbb{Z}_p^*$, 
where $p$ is prime. Also, we write $Z_{q}[X]_d$ to denote the set of all $d$ degree 
polynomials univariate in $X$ with coefficients in the finite field $\mathbb{Z}_q$. 

\subsection{Packed Shamir Secret Sharing}
\label{sec:packed-shamir}
$(n,t,\ell)$-Packed Shamir secret sharing (\cite{10.1145/129712.129780},\cite{crypto-1984-905})
 scheme is a threshold secret sharing scheme which is a variant of $(n,t)$-Shamir's 
 secret sharing scheme \cite{10.1145/359168.359176}. In a nutshell, the $t+\ell-1$ 
 degree secret polynomial with coefficients in $\mathbb{Z}_q$ which evaluates to $\ell$ 
 secrets is secret shared amongst $n$ parties such that any $t+\ell$ parties can 
 reconstruct back the secret polynomial. Recall that Shamir's secret sharing scheme 
 requires at least $t+1$ parties to reconstruct the secret polynomial in contrast to 
 the $t+\ell$ parties in the Packed Shamir secret sharing scheme. The scheme is
 summarized in the Figure \ref{fig:packed-shamir}.
\begin{figure}[ht]
    \centering
    \begin{tcolorbox}[title=\textbf{Packed Shamir Secret Sharing}, width=0.9\textwidth, colframe=blue!75!black, colback=blue!10, sharp corners]
        Given $\ell$ secrets to share amongst $n$ parties, where at most $t$ of them can
        be \textit{(passively)} corrupt, the $(n,t,\ell)$-Packed Shamir secret sharing scheme description is as
        follows:
        
        \vspace{0.5em}
        \textbf{Sharing Algorithm:}
        \begin{itemize}
            \item Dealer constructs the secret polynomial $f\in\mathbb{Z}_q[X]_{t+l-1}$
                  via the lagrange interpolation by choosing $t+\ell$ elements in 
                  $\mathbb{Z}_q$ where $\ell$ of them are secrets, $\{s_i\}_{i=0}^{\ell-1}$, with
                  $f(-i)=s_i$ for all $i$ and remaining $t$ are chosen uniformly 
                  at random in $\mathbb{Z}_q$.
            \item Each party $P_i$ receives their share $f(i)$ from the Dealer
                  for each $i\in\{1,\dots,n\}$
        \end{itemize}
        
        \vspace{0.5em}
        \textbf{Reconstruction Algorithm:}
        \begin{itemize}
            \item Any $\mathcal{Q}$ set containing at least $t+\ell$ parties can use the 
            lagrange interpolation to compute $\{s_i\}_{i=0}^{\ell-1}$ as follows:
            \begin{align*}
                s_m &= \sum_{i\in \mathcal{Q}} f(i) \left[\prod_{j\in \mathcal{Q}, j\neq i}\frac{-m-j}{i-j}\right] &&, m\in\{0,\dots,\ell-1\} \\
            \end{align*}
            \item The secrets $\{s_i\}_{i=0}^{\ell-1}$ are outputted as the result.
        \end{itemize}
    \end{tcolorbox}
    \caption{Packed Shamir Secret Sharing}
    \label{fig:packed-shamir}
\end{figure}

\subsection{Sigma Protocols}
\label{sec:sigma-protocols}
The agenda of this subsection is to give a brief formal background about some important primitives 
used in the PVSS ,$\Pi_S$ \cite{cryptoeprint:2023/1669}, and the PPVSS ,$\Lambda_{RO}$ \cite{cryptoeprint:2025/576}, schemes.
Let $X$ and $W$ be two sets with $R$ being a relation on $X\times W$, and $L=\{x\in X :\exists w\in W, xRw\}$
be the language defined by $R$ where $xRw$ says that $w$ is a witness for a given $x\in L$. 
Also, let $\mathcal{R}$ be a PPT algorithm such that $\mathcal{R}(1^\lambda)$ outputs pairs 
$(x,w)$ with $x\in L$ and $xRw$ where $\lambda$ is a security parameter.\par

Given a relation $R$ and its corresponding language $L$, a \textbf{Sigma ($\sum$) Protocol} 
is a $3$-round \textit{interactive} protocol between two Probabilistic Polynomial Time (PPT) algorithms, 
a prover $P$ 
and a verifier $V$. For some $x\in L$ with $xRw$, in the first round $P$ sends a 
commitment $a$ to $V$. To which $V$ sends a challenge $d$ to $P$ in the second round 
and finally $P$ responds back with the response $z$ to $V$ in the third round. 
$V$ outputs \textbf{true} or \textbf{false} upon the proof verification on transcript
$trans := (a, d, z)$. 
Informally, with a $\sum-$protocol a prover $P$ tries to convince
a verifier $V$ that they know a witness $w$ for a given statement $x\in L$ without 
revealing any information about $w$. To state it formally, a $\sum-$protocol is 
supposed to satisfy \textit{completeness}, \textit{Honest Verifier Zero Knowledge} (HVZK)
and \textit{Special Soundness} which are defined as follows.

\begin{definition}[Completeness]
  A $\sum-$protocol is said to be \textbf{complete} for $\mathcal{R}$ if
  the verifier $V$ always accepts the honest prover $P$ for any $x\in L$.
\end{definition}

\begin{definition}[HVZK]
  A $\sum-$protocol is said to be \textbf{HVZK} for $\mathcal{R}$ if there exist a PPT algorithm $S$ 
  that simulates $trans$ of the scheme corresponding to a given $x\in L$ with any witness $w$ of $x$. 
  That is, given $x\in L$,
  \begin{align*}
    trans(P(x,w)\leftrightarrow V(x)) &\approx trans(S(x) \leftrightarrow V(x))&& \text{, for any witness $w$ of $x$.}
  \end{align*}
  Where $trans(P(\cdot)\leftrightarrow V(\cdot))$ is the transcript of the $\sum-$protocol amongst 
  $P$ and $V$ and $\approx$ denotes the indistinguishability of the two transcripts.
\end{definition}

\begin{definition}[Special Soundness]
  A $\sum-$protocol is said to satisfy \textbf{Special Soundness} for $\mathcal{R}$, 
  if there exists a PPT extractor $\mathcal{E}$ for any two valid transcripts, $(a,d,z)$ and 
  $(a,d',z')$, corresponding to a given $x\in L$ with only a unique witness $w$ and $d\neq d'$ such that $\mathcal{E}(a,d,z,d',z')$ outputs the 
  witness $w$. 
\end{definition}

It is shown that a public-coin, complete, HVZK, special soundness $\sum-$protocol can be made into a
Non Interactive Zero Knowledge (NIZK) Proof of Knowledge (PoK) or Argument of Knowledge (AoK) in the 
Random Oracle($RO$) model using Fiat-Shamir transform \cite{10.1007/3-540-47721-7_12}. 
In the following subsections, we recall two important NIZK PoK schemes which are used in $\Pi_S$ and 
$\Lambda_{RO}$ schemes.

\subsection{Chaum-Pedersen Protocol for DL Equality}
Consider $\mathbb{G}$ being the cyclic group of prime order $q$ with hard Discrete Logarithm (DL). 
For some $g,h\in \mathbb{G}$ consider the following relation:
\begin{align*}
  R_{DLEQ} &= \{(g,h,a,b),x : a=g^x, b=h^x\}.
\end{align*}
In \cite{10.1007/3-540-48071-4_7}, Chaum and Pedersen proposed a NIZK PoK scheme for the DL Equality 
relation, $R_{DLEQ}$. Informally, a prover $P$ can convince a verifier $V$ that they know $x$ such that
it can be used with both $g$ and $h$ to obtain $a$ and $b$ respectively. This protocol is widely used in
many cryptographic applications like threshold decryption, e-voting and Randomness Beacons. 
We summarize the protocol in Figure \ref{fig:chaum-pedersen}.
\begin{figure}[ht]
    \centering
    \begin{tcolorbox}[title=\textbf{Chaum-Pedersen Protocol for DLEQ}, width=0.9\textwidth, colframe=blue!75!black, colback=blue!10, sharp corners]
        Let $(g,h,a,b)\in L_{DLEQ}$ be a statement with its corresponding witness being $x$ where $L_{DLEQ}$
        is the language defined by the relation $R_{DLEQ}$ and $\mathcal{H}$ be a Random Oracle ($RO$).
        
        \vspace{0.5em}
        \textbf{Prover}
        \begin{itemize}
            \item Samples $r\in_{R}\mathbb{Z}_q$ uniformly at random and sets 
                $c_1=g^r$ and $c_2=h^r$.
            \item Sets $d\leftarrow \mathcal{H}(a,b,c_1,c_2)$, where $\mathcal{H}$ is 
                an agreed upon Random Oracle ($RO$).
            \item Sets $z\equiv r+dx \pmod{q}$ and returns the proof(/transcript) $\pi:= (d,z)$.
        \end{itemize}
        
        \vspace{0.5em}
        \textbf{Verifier}
        \begin{itemize}
            \item Checks if $d\leftarrow \mathcal{H}(a,b,\frac{g^z}{a^d},\frac{h^z}{b^d})$ 
                and outputs \textbf{true} or \textbf{false} accordingly.
        \end{itemize}
    \end{tcolorbox}
    \caption{Chaum-Pedersen NIZK PoK for DLEQ}
    \label{fig:chaum-pedersen}
\end{figure}


\section{The First Topic of the Chapter}
First comes the introduction to this topic.ufyufty

\lipsum[55]

\subsection{An item}
Please don't abuse enumerations: short enumerations shouldn't use
``\verb|itemize|'' or ``\texttt{enumerate}'' environments.
So \emph{never write}: 
\begin{quote}
  The Eiffel tower has three floors:
  \begin{itemize}
  \item the first one;
  \item the second one;
  \item the third one.
  \end{itemize}
\end{quote}
But write:
\begin{quote}
  The Eiffel tower has three floors: the first one, the second one, and the
  third one.
\end{quote}

\section{A Second Topic}
\lipsum[64]

\subsection{Another item}
\lipsum[56-57]

\section{Conclusion}
The final section of the chapter gives an overview of the important results
of this chapter. This implies that the introductory chapter and the
concluding chapter don't need a conclusion.

\lipsum[66]

%%% Local Variables: 
%%% mode: latex
%%% TeX-master: "thesis"
%%% End: 
