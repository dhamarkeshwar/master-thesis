\chapter{Preliminaries}
\label{chap:preliminaries}
\section{Notation}
Let $\mathbb{G}$ be a cyclic group of prime order $q$ with hard Discrete Log (DL) and its generator 
being $g$. 
% isomorphic to a subgroup of the multiplicative modular group $\mathbb{Z}_p^*$, where $p$ is prime. 
Also, we write $\mathbb{Z}_{q}[X]_d$ to denote the set of all $d$ degree 
polynomials univariate in $X$ with coefficients in the finite field $\mathbb{Z}_q$. 

\section{Coding Theory}
\label{sec:linear-codes}
This subsection is a brief recall of linear codes and their properties.

\begin{definition}[Linear Code]
  If $\mathcal{C}$ be a vector subspace of $\mathbb{Z}_q^n$ with dimension $k$, then $\mathcal{C}$ is 
  said to be a \textbf{linear code}(/ linear $q-$ary code) of length $n$ and dimension $k$.
\end{definition}

In the remainder of the subsection, we let $\mathcal{C}$ be a linear $q-$ary code of length $n$ and 
dimension $k$.

\begin{definition}[Dual Code]
  The vector subspace $\mathcal{C}^{\perp}$ is called a Dual (Code) of $\mathcal{C}$ if it is 
  orthogonal to $\mathcal{C}$.
\end{definition}

\begin{definition}[Generating Matrix]
  The $k\times n-$matrix $\mathcal{G}$ is said to be a generating matrix of $\mathcal{C}$ if it 
  generates $\mathcal{C}$, more precisely, the rows of $G$ form a basis for $\mathcal{C}$. Also, $\mathcal{G}$ is said to be in its 
  \textbf{standard form} if it is of the form
  \begin{align*}
    \mathcal{G} = \begin{bmatrix}
      I_k & P
    \end{bmatrix},
  \end{align*}
  where $I_k$ is the $k\times k$ identity matrix and $P$ is some $k\times (n-k)$ matrix.
\end{definition}

\begin{definition}[Parity Check Matrix]
  Consider the linear transformation $\phi$ as follows:
  \begin{align*}
    \phi:& \mathbb{Z}_q^n \rightarrow \mathbb{Z}_q^{n-k},
  \end{align*}
  where kernel of $\phi$ is $\mathcal{C}$. Then the matrix associated to $\phi$, $\mathcal{H}$, 
  is called the \textbf{parity check matrix} of $\mathcal{C}$.
\end{definition}

\begin{lemma}
  If $\mathcal{G}$ being a generating matrix of $\mathcal{C}$ and say it is in its standard form, i.e., $\mathcal{G} = \begin{bmatrix}
    I_k & P
  \end{bmatrix}$, then $\mathcal{H}$ being a parity check matrix of $\mathcal{C}$ is given by
  \begin{align*}
    \mathcal{H} = \begin{bmatrix}
      -P^T & I_{n-k}
    \end{bmatrix},
  \end{align*}
  where $I_{n-k}$ is the $(n-k)\times (n-k)$ identity matrix and $P^T$ is the transpose of $P$.
\end{lemma}

\subsection{Reed Solomon Codes}
\label{subsec:reed-solomon}


\section{Packed Shamir Secret Sharing}
\label{sec:packed-shamir}
$(n,t,\ell)$-Packed Shamir secret sharing (\cite{10.1145/129712.129780},\cite{crypto-1984-905})
 scheme is a threshold secret sharing scheme which is a variant of $(n,t)$-Shamir's 
 secret sharing scheme \cite{10.1145/359168.359176}. In a nutshell, the $t+\ell-1$ 
 degree secret polynomial with coefficients in $\mathbb{Z}_q$ which evaluates to $\ell$ 
 secrets is secret shared amongst $n$ parties such that any $t+\ell$ parties can 
 reconstruct back the secret polynomial. Recall that Shamir's secret sharing scheme 
 requires at least $t+1$ parties to reconstruct the secret polynomial in contrast to 
 the $t+\ell$ parties in the Packed Shamir secret sharing scheme. The scheme is
 summarized in the Figure \ref{fig:packed-shamir}.
\begin{figure}[ht]
    \centering
    \begin{tcolorbox}[title=\textbf{Packed Shamir Secret Sharing}, width=0.9\textwidth, colframe=blue!75!black, colback=blue!10, sharp corners]
        Given $\ell$ secrets to share amongst $n$ parties, where at most $t$ of them can
        be \textit{(passively)} corrupt, the $(n,t,\ell)$-Packed Shamir secret sharing scheme description is as
        follows:
        
        \vspace{0.5em}
        \textbf{Sharing Algorithm:}
        \begin{itemize}
            \item Dealer constructs the secret polynomial $f\in\mathbb{Z}_q[X]_{t+l-1}$
                  via the lagrange interpolation by choosing $t+\ell$ elements in 
                  $\mathbb{Z}_q$ where $\ell$ of them are secrets, $\{s_i\}_{i=0}^{\ell-1}$, with
                  $f(-i)=s_i$ for all $i$ and remaining $t$ are chosen uniformly 
                  at random in $\mathbb{Z}_q$.
            \item Each party $P_i$ receives their share $f(i)$ from the Dealer
                  for each $i\in\{1,\dots,n\}$
        \end{itemize}
        
        \vspace{0.5em}
        \textbf{Reconstruction Algorithm:}
        \begin{itemize}
            \item Any $\mathcal{Q}$ set containing at least $t+\ell$ parties can use the 
            lagrange interpolation to compute $\{s_i\}_{i=0}^{\ell-1}$ as follows:
            \begin{align*}
                s_m &= \sum_{i\in \mathcal{Q}} f(i) \left[\prod_{j\in \mathcal{Q}, j\neq i}\frac{-m-j}{i-j}\right] &&, m\in\{0,\dots,\ell-1\} \\
            \end{align*}
            \item The secrets $\{s_i\}_{i=0}^{\ell-1}$ are outputted as the result.
        \end{itemize}
    \end{tcolorbox}
    \caption{Packed Shamir Secret Sharing}
    \label{fig:packed-shamir}
\end{figure}

\section{Sigma Protocols}
\label{sec:sigma-protocols}
The agenda of this subsection is to give a brief formal background about some important primitives 
used in the PVSS ,$\Pi_S$ \cite{cryptoeprint:2023/1669}, and the PPVSS ,$\Lambda_{RO}$ \cite{cryptoeprint:2025/576}, schemes.
Let $X$ and $W$ be two sets with $R$ being a relation on $X\times W$, and $L=\{x\in X :\exists w\in W, xRw\}$
be the language defined by $R$ where $xRw$ says that $w$ is a witness for a given $x\in L$. 
Also, let $\mathcal{R}$ be a PPT algorithm such that $\mathcal{R}(1^\lambda)$ outputs pairs 
$(x,w)$ with $x\in L$ and $xRw$ where $\lambda$ is a security parameter.\par

Given a relation $R$ and its corresponding language $L$, a \textbf{Sigma ($\sum$) Protocol} 
is a $3$-round \textit{interactive} protocol between two Probabilistic Polynomial Time (PPT) algorithms, 
a prover $P$ 
and a verifier $V$. For some $x\in L$ with $xRw$, in the first round $P$ sends a 
commitment $a$ to $V$. To which $V$ sends a challenge $d$ to $P$ in the second round 
and finally $P$ responds back with the response $z$ to $V$ in the third round. 
$V$ outputs \textbf{true} or \textbf{false} upon the proof verification on transcript
$trans := (a, d, z)$. 
Informally, with a $\sum-$protocol a prover $P$ tries to convince
a verifier $V$ that they know a witness $w$ for a given statement $x\in L$ without 
revealing any information about $w$. To state it formally, a $\sum-$protocol is 
supposed to satisfy \textit{completeness}, \textit{Honest Verifier Zero Knowledge} (HVZK)
and \textit{Special Soundness} which are defined as follows.

\begin{definition}[Completeness]
  A $\sum-$protocol is said to be \textbf{complete} for $\mathcal{R}$ if
  the verifier $V$ always accepts the honest prover $P$ for any $x\in L$.
\end{definition}

\begin{definition}[HVZK]
  A $\sum-$protocol is said to be \textbf{HVZK} for $\mathcal{R}$ if there exist a PPT algorithm $S$ 
  that simulates $trans$ of the scheme corresponding to a given $x\in L$ with any witness $w$ of $x$. 
  That is, given $x\in L$,
  \begin{align*}
    trans(P(x,w)\leftrightarrow V(x)) &\approx trans(S(x) \leftrightarrow V(x))&& \text{, for any witness $w$ of $x$.}
  \end{align*}
  Where $trans(P(\cdot)\leftrightarrow V(\cdot))$ is the transcript of the $\sum-$protocol amongst 
  $P$ and $V$ and $\approx$ denotes the indistinguishability of the two transcripts.
\end{definition}

\begin{definition}[Special Soundness]
  A $\sum-$protocol is said to satisfy \textbf{Special Soundness} for $\mathcal{R}$, 
  if there exists a PPT extractor $\mathcal{E}$ for any two valid transcripts, $(a,d,z)$ and 
  $(a,d',z')$, corresponding to a given $x\in L$ with only a unique witness $w$ and $d\neq d'$ such that $\mathcal{E}(a,d,z,d',z')$ outputs the 
  witness $w$. 
\end{definition}

It is shown that a public-coin, complete, HVZK, special soundness $\sum-$protocol can be made into a
Non Interactive Zero Knowledge (NIZK) Proof of Knowledge (PoK) or Argument of Knowledge (AoK) in the 
Random Oracle ($RO$) model using Fiat-Shamir transform \cite{10.1007/3-540-47721-7_12}. 
In the following subsections, we recall two important NIZK PoK schemes which are used in $\Pi_S$ and 
$\Lambda_{RO}$ schemes.

\subsection{Chaum-Pedersen Protocol for DL Equality}
\label{subsec:chaum-pedersen}
Recall $\mathbb{G}$ being the cyclic group of prime order $q$ with hard Discrete Logarithm (DL). 
For some $g,h\in \mathbb{G}$ consider the following relation:
\begin{align*}
  R_{DLEQ} &= \{(g,h,a,b),x : a=g^x, b=h^x\}.
\end{align*}
In \cite{10.1007/3-540-48071-4_7}, Chaum and Pedersen proposed a NIZK PoK scheme for the DL Equality 
relation, $R_{DLEQ}$. Informally, a prover $P$ can convince a verifier $V$ that they know $x$ such that
it can be used with both $g$ and $h$ to obtain $a$ and $b$ respectively. This protocol is widely used in
many cryptographic applications like threshold decryption, e-voting and Randomness Beacons. 
We summarize the protocol in Figure \ref{fig:chaum-pedersen}.
\begin{figure}[ht]
    \centering
    \begin{tcolorbox}[title=\textbf{Chaum-Pedersen Protocol for DLEQ}, width=0.9\textwidth, colframe=blue!75!black, colback=blue!10, sharp corners]
        % Let $(g,h,a,b)\in L_{DLEQ}$ be a statement with its corresponding witness being $x$ where $L_{DLEQ}$
        % is the language defined by the relation $R_{DLEQ}$ and 
        Let $g$ be the group generator sampled in the setup phase, $[(g,h,a,b),x]\in R_{DLEQ}$ and 
        $\mathcal{H}$ be a Random Oracle ($RO$).\par
        \vspace{0.5em}
        \textbf{Prover}
        \begin{itemize}
            \item Samples $r\in_{R}\mathbb{Z}_q$ uniformly at random and sets 
                $c_1=g^r$ and $c_2=h^r$.
            \item Sets $d\leftarrow \mathcal{H}(a,b,c_1,c_2)$, where $\mathcal{H}$ is 
                an agreed upon Random Oracle ($RO$).
            \item Sets $z\equiv r+dx \pmod{q}$ and returns the proof(/transcript) $\pi:= (d,z)$.
        \end{itemize}
        
        \vspace{0.5em}
        \textbf{Verifier}
        \begin{itemize}
            \item Checks if $d\leftarrow \mathcal{H}(a,b,\frac{g^z}{a^d},\frac{h^z}{b^d})$ 
                and outputs \textbf{true} or \textbf{false} accordingly.
        \end{itemize}
    \end{tcolorbox}
    \caption{Chaum-Pedersen NIZK PoK for DLEQ}
    \label{fig:chaum-pedersen}
\end{figure}


\subsection{NIZK PoK for Polynomial DL}
\label{subsec:polynomial-dl}
Recall $\mathbb{G}$ being the cyclic group of prime order $q$ with hard Discrete Logarithm (DL) and $g$ 
being its generator. Consider the following relation for some polynomial $f\in\mathbb{Z}_q[X]_t$ with
degree $t<n$:
\begin{align*}
  R_{PDL} &= \{(g,x_1,\dots,x_n,F(x_1),\dots,F(x_n)),f(X) : F(x_i)=g^{f(x_i)}, 1\leq i\leq n\}.
\end{align*}
In \cite{cryptoeprint:2023/1669}, Baghery formally introduced a NIZK PoK scheme for the Polynomial DL 
relation, $R_{PDL}$, which is a generalization of Schnorr's ID protocol \cite{crypto-1989-1727}. Informally, 
a prover $P$ can convince a verifier $V$ that they know a $t$ degree polynomial $f$ such that 
it can be used with $g$ to obtain $F(x_i)$ for $1\leq i\leq n$. This protocol is used to construct the PPVSS 
$\Lambda_{RO}$ \cite{cryptoeprint:2025/576}, which was essential in building an efficient e-voting protocol. 
We summarize the protocol in Figure \ref{fig:polynomial-dl}.
\begin{figure}[ht]
    \centering
    \begin{tcolorbox}[title=\textbf{A NIZK PoK for Polynomial DL}, width=0.9\textwidth, colframe=blue!75!black, colback=blue!10, sharp corners]
        Let $(g,x_1,\dots,x_n,F(x_1),F(x_n))\in L_{PDL}$ be a statement with its corresponding witness being $f\in\mathbb{Z}_q[X]_t$ 
        where $L_{PDL}$ is the language defined by the relation $R_{PDL}$.
        
        \vspace{0.5em}
        \textbf{Prover}
        \begin{itemize}
            \item Samples $r\in_{R}\mathbb{Z}_q[X]_t$ uniformly at random and sets 
                $\Gamma_i=g^{r(x_i)}$ for $1\leq i\leq n$.
            \item Sets $d\leftarrow \mathcal{H}(F_1, \dots, F_n, \Gamma_1, \dots, \Gamma_n)$, where $\mathcal{H}$ is 
                an agreed upon Random Oracle (RO).
            \item Sets $z(X)\equiv r(X)+df(X) \pmod{q}$ and returns the proof(/transcript) $\pi:= (d,z(X))$.
        \end{itemize}
        
        \vspace{0.5em}
        \textbf{Verifier}
        \begin{itemize}
            \item First checks if $z$ is a $t$ degree polynomial in $\mathbb{Z}_q[X]_t$. If so, they proceed with the next step.
            \item Checks if $d\leftarrow \mathcal{H}(F_1, \dots, F_n,\frac{g^{z(x_1)}}{F_1^d}, \dots, \frac{g^{z(n)}}{F_n^d})$. 
            \item If first two steps are correct then they output \textbf{true}, otherwise \textbf{false}.
        \end{itemize}
    \end{tcolorbox}
    \caption{A NIZK PoK for Polynomial DL based on Schoenmakers' PVSS}
    \label{fig:polynomial-dl}
\end{figure}


\section{Publicly Verifiable Secret Sharing (PVSS)}
\label{sec:pvss}
Publicly Verifiable Secret Sharing (PVSS) is an extension of Non-Interactive Verifiable Secret Sharing 
(NI-VSS) scheme. Unlike NI-VSS where only the parties who possess the secret shares can verify the 
correctness of the secret sharing, anyone including external entities can verify the correctness of 
the secret sharing in PVSS. 

\section{Pre-Constructed Publicly Verifiable Secret Sharing (PPVSS)}
\label{sec:ppvss}
PPVSS was first introduced in \cite{cryptoeprint:2025/576}, which is used as a building block to 
construct a new e-voting protocol based on Schoenmakers' PVSS \cite{5581ccd9530540479539d21d1d39ae96}. 
Interestingly, the authors in \cite{cryptoeprint:2025/576} observed that the original e-voting protocol 
published in 1999 by Schoenmakers is unusually efficient to be just based on a PVSS, which led them to 
discover that Schoenmakers PVSS is actually a PPVSS. What sets PPVSS apart from standard PVSS schemes 
is that it can be used to construct versatile applications, such as e-voting, and can also improve 
efficiency of some existing protocols. The subtle difference between PPVSS and PVSS is that the secret 
itself is committed by the prover along with all its corresponding secret shares.

\section{Conclusion}
The final section of the chapter gives an overview of the important results
of this chapter. This implies that the introductory chapter and the
concluding chapter don't need a conclusion.

%%% Local Variables: 
%%% mode: latex
%%% TeX-master: "thesis"
%%% End: 
