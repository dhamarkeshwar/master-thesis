\chapter{Literature Review}
\label{cha:0}
In 1979, Shamir introduced a threshold secret sharing scheme called 
Shamir Secret Sharing scheme \cite{10.1145/359168.359176}, which is now a well-known 
and widely used secret sharing scheme to this day because of its numerous applications in
cryptography. It was first of its kind to have Information Theoretic (IT) security 
under certain assumptions against passive adversaries who can only see the secret 
shares of the parties they have corrupted. In reality, however, the adversaries 
are usually stronger than just being passive, moreover, they possess the power 
to manipulate the share values of the corrupted parties itself. 
Shamir's scheme is not tailored to defend against active adversaries as one cannot 
verify the correctness of the shares. This led to numerous inventions of Verifiable 
Secret Sharing (VSS) schemes, which not only does allow the parties to verify the 
correctness of the shares shared by the dealer but also allows the parties to verify 
the correctness of the shares when opened by the parties during the reconstruction 
phase. Because of the feature of verifiability, VSS schemes can defend the applications 
against active adversaries.\par

There are many VSS schemes (\cite{d053b0be49644b2f932d703db8c1f8a0}, \cite{DBLP:conf/focs/Feldman87}) 
in the literature which are based on Shamir Secret Sharing scheme. Throughout the years, many advancements 
have been made in the field of VSS schemes, and as of writing this report the efficient VSS schemes are 
$\Pi_F$, $\Pi_P$ and $\Pi_{LA}$ \cite{cryptoeprint:2023/1669}, each of which have distinct security features. 
In VSS, only shareholders can actually verify the correctness of the shares. Certain applications demand 
to have verifiability feature available to anyone, which is solved by Publicly Verifiable Secret Sharing (PVSS) 
schemes. PVSS is an extension of VSS, where the correctness of the shares can be verified by anyone. Many 
cool applications exist today which use PVSS schemes, such as, e-voting \cite{5581ccd9530540479539d21d1d39ae96}, 
randomness beacons \cite{cryptoeprint:2017/216}, etc. In \cite{cryptoeprint:2025/576}, authors have noticed 
that the Schoenmakers' PVSS scheme used for the e-voting application in \cite{5581ccd9530540479539d21d1d39ae96} 
is actually more than a PVSS scheme, and they coined the term Pre-Constructed Publicly Verifiable Secret Sharing (PPVSS) scheme. 
PPVSS is a special type of PVSS where the dealer additionally publishes a commitment to the secret itself. 
The authors have also shown that any PVSS scheme can be transformed into a PPVSS scheme with minimal 
changes, and constructed a PPVSS $\Lambda_{RO}$ from the PVSS $\Pi_S$ \cite{cryptoeprint:2023/1669} as an 
example, where they used $\Lambda_{RO}$ to build an efficient e-voting application.\par 

With PPVSS, one can build versatile applications and also can improve the efficiency of existing 
applications. In ALBATROSS \cite{cryptoeprint:2020/644}, authors built a randomness beacon application 
using a PVSS. We have an intuition that an efficient randomness beacon application can be built 
using a scheme based on PPVSS on certain conditions. In this report, we will introduce Packed PPVSS (PPPVSS) 
along with its security proofs and give an example based on $\Lambda_{RO}$, which will be used to improve 
ALBATROSS in many cases.\par

% In this chapter we sequently recall Packed Shamir secret sharing, 
% Sigma ($\sum$) Protocols and Publicly Verifiable Secret Sharing (PVSS) followed by
% the recent scheme introduced in \cite{cryptoeprint:2025/576}, namely, 
% Pre-Constructed Publicly Verifiable Secret Sharing (PPVSS) which has versatile 
% applications and also improves efficiency in existing applications.
% The agenda of this chapter is to give enough background before describing our Packed PPVSS (PPPVSS) scheme 
% and its corresponding security guarantees in the next chapter.

%%% Local Variables: 
%%% mode: latex
%%% TeX-master: "thesis"
%%% End: 
