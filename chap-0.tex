\chapter{Introduction}
\label{cha:0}

In this rapidly evolving digital world, cryptography was and is playing a crucial role, which 
allows a user to securely communicate and process sensitive information without any 
fear of eavesdropping or tampering the data. Their whole trust on cryptography boils 
down to their secret keys, which are used to encrypt and/or decrypt their data. In 
reality, however, these secret keys are stored as a whole in their device, making it 
a single point of failure such that an adversary can just steal their secret keys. 
This seriously affects the robustness and availability of the services offered by 
such digital infrastructures. A trivial solution for availability aspect is to make 
multiple copies of the secret keys and store them in multiple devices. However, 
this is not a good idea because now an adversary has multiple points to attack leading 
to decrease in the robustness of the system security. Now one can possibly think to 
divide a given secret key into multiple pieces such that any single piece or up to a 
threshold number of pieces can reveal nothing about the secret key.\par 

In 1979, Shamir introduced a threshold secret sharing scheme called 
Shamir Secret Sharing scheme \cite{10.1145/359168.359176}, which is now a well-known 
and widely used secret sharing scheme to this day because of its numerous applications in
cryptography. Basically, it allows a person to divide their secret into many pieces and distribute 
those pieces amongst other entities, such that a single entity or up to a group of 
entities cannot determine anything about the secret. To reconstruct the secret back, 
one would require more than a threshold number of entity shares. Importantly and interestingly, 
under certain conditions it is proven to be 
secure against passive adversaries who can see secret shares of some parties and also have 
unlimited computational power.\par 

Shamir secret sharing scheme was first of its kind to have such Information Theoretic (IT) security, 
under certain assumptions, against such passive adversaries. In reality, however, the adversaries 
are usually stronger than just being passive, moreover, they can be active where they 
possess the power to manipulate the share values of the corrupted parties itself. For example, 
an active adversary can manipulate some secret key shares then reconstruction protocol 
for the secret key may yield a wrong secret key, which will badly affect the robustness 
of the system. As Shamir's scheme is not tailored to defend against active adversaries as one cannot 
verify the correctness of the shares, it led to inventing Verifiable 
Secret Sharing (VSS) schemes, which not only does allow the parties to verify the 
correctness of the shares shared by the dealer but also allows the parties to verify 
the correctness of the shares when revealed by the parties during the reconstruction 
phase. This allows VSS schemes to pin point who cheated during the protocol unlike 
in Shamir's scheme which cannot feasibly achieve such functionality. 
Because of the feature of verifiability, VSS schemes found their way into 
many applications which require security against active adversaries.\par

There are many VSS schemes (\cite{d053b0be49644b2f932d703db8c1f8a0}, \cite{DBLP:conf/focs/Feldman87}) 
in the literature which are based on Shamir secret sharing scheme. Throughout the years, many advancements 
have been made in the field of VSS schemes, and as of writing this report the efficient VSS schemes are 
$\Pi_F$, $\Pi_P$ and $\Pi_{LA}$ \cite{cryptoeprint:2023/1669}, each of which have distinct security features. 
In VSS, only shareholders can actually verify the correctness of the shares. Certain applications demand 
to have verifiability feature available to anyone, such is offered by Publicly Verifiable Secret Sharing (PVSS) 
schemes. PVSS is an extension of VSS, where the correctness of the shares can be verified by anyone. Many 
cool applications exist today which use PVSS schemes, such as, e-voting \cite{5581ccd9530540479539d21d1d39ae96}, 
randomness beacons \cite{cryptoeprint:2017/216}, etc. In \cite{cryptoeprint:2025/576}, authors have noticed 
that the Schoenmakers' PVSS scheme used for the e-voting application in \cite{5581ccd9530540479539d21d1d39ae96} 
is actually more than a PVSS scheme, and they coined the term Pre-Constructed Publicly Verifiable Secret Sharing (PPVSS) scheme. 
PPVSS is a special type of PVSS where the dealer additionally publishes a commitment (/encryption) to the secret itself. 
The authors have also shown that any PVSS scheme can be transformed into a PPVSS scheme with minimal 
changes, and constructed a PPVSS $\Lambda_{RO}$ from the PVSS $\Pi_S$ \cite{cryptoeprint:2023/1669} and 
used it to build an efficient e-voting application.\par 

With PPVSS, one can build versatile applications and also can improve the efficiency of existing 
applications. For instance, in ALBATROSS \cite{cryptoeprint:2020/644} authors built a randomness beacon application 
using a PVSS. We have an intuition that an efficient randomness beacon application can be built 
using a scheme based on PPVSS on certain conditions. In this thesis, we will introduce Packed PPVSS (PPPVSS or 3PVSS), 
an extension of PPVSS where shares representing multiple secrets are secret shared,  
along with its security proofs and give an example based on $\Lambda_{RO}$, which will be used to improve 
ALBATROSS in many cases.\par

The research work presented in this thesis is based on the following questions:
\begin{itemize}
    \item \textit{What more functionalities can be achieved through PPVSS schemes?}
    \item \textit{How to generalize the PPVSS schemes to the packed version which allows to 
      secret share multiple secrets efficiently?}
    \item \textit{Can we improve the efficiency of some existing applications using PPVSS schemes
      without compromising their security aspects?}
\end{itemize}


\section*{Outline of the thesis}
The next chapter [\ref{chap:preliminaries}] gives the necessary preliminaries required 
to understand the mathematical security guarantees of packed Shamir secret sharing, sigma 
protocols and some examples that were used to build some PVSS and PPVSS schemes. The chapter 
ends with the description of PPVSS scheme and a realtime example of the same. Moving forward, 
chapter \ref{cha:3} introduces packed PPVSS (PPPVSS) scheme and two examples along with 
their security proofs where the proof for latter example is a bit non-trivial. The 
chapter \ref{cha:n} presents a new randomness beacon protocol based on a PPPVSS and 
compares it with the state-of-the-art randomness beacon protocol.
Finally, the chapter \ref{cha:conclusion} concludes the thesis with a summary of the 
results and discusses the possible future work and applications of the PPPVSS scheme.

% In this chapter we sequently recall Packed Shamir secret sharing, 
% Sigma ($\sum$) Protocols and Publicly Verifiable Secret Sharing (PVSS) followed by
% the recent scheme introduced in \cite{cryptoeprint:2025/576}, namely, 
% Pre-Constructed Publicly Verifiable Secret Sharing (PPVSS) which has versatile 
% applications and also improves efficiency in existing applications.
% The agenda of this chapter is to give enough background before describing our Packed PPVSS (PPPVSS) scheme 
% and its corresponding security guarantees in the next chapter.

%%% Local Variables: 
%%% mode: latex
%%% TeX-master: "thesis"
%%% End: 
