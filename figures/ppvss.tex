\begin{figure}[ht]
    \centering
    \resizebox{\textwidth}{!}{ % Further reduced the scaling factor to fit the content
    \begin{tcolorbox}[title=\textbf{$\Lambda_{RO}$ \cite{cryptoeprint:2025/576}}, width=1.2\textwidth, colframe=blue!75!black, colback=blue!10, sharp corners]
        
        \textbf{Initialization:}
            All parties $\{P_i\}_{i=1}^n$ and dealer $D$ agree on the prime field $\mathbb{Z}_q$, a group $(\mathbb{G},\times)$ of 
            order $q$ with a generator $g$, random oracle $\mathcal{H}$. Also, each party 
            $P_i$ registers their public key $PK_i$ in the public ledger and all agree on a commitment 
            key or a public key, $PK_0\neq g$, whose secret key is known to a target person.

        \vspace{0.5em}
        \textbf{Share:}
        \begin{itemize}
            \item Dealer $D$ samples a $t$-degree polynomial $f\in\mathbb{Z}_q[X]$ uniformly at random and 
              sets $g^{f_0}$ as the secret to be shared, where $f_0=f(0)$.
            \item For each $1\leq i\leq n$, $D$ computes $f(i)=f_i$ and uses $PK_i$ to encrypt the secret share  
               to obtain $PK_i^{f_i}=y_i$. $D$ also encrypts(/commits) to the secret $g^{f_0}$ using the encryption(/commitment) 
              key $PK_0$ to obtain $y_0=PK_0^{f_0}$. 
            \item $D$ uses the PDL proof scheme \ref{subsec:polynomial-dl} to generate $\pi_{PDL}$ as follows:
                 \begin{itemize}
                    \item Samples a $t-$degree polynomial $r\in\mathbb{Z}_q[X]$ uniformly at random and 
                    computes $c_i=PK_i^{r_i}$ where $r_i=r(i)$ for $0\leq i\leq n$.
                    \item Using $\mathcal{H}$, $d=\mathcal{H}(y_0,\dots,y_n,c_0,\dots,c_n)$ is computed.
                    \item Sets $z(X)=r(X)+df(X)$, hence $\pi_{PDL}=(d,z(X))$ is obtained. 
                 \end{itemize}
            \item $D$ broadcasts the encryptions of the shares along with the encryption(/commitment) of the secret 
              with $\pi_{PDL}$ which proves the validity of the encrypted shares and encrypted(/committed) 
              secret, i.e., broadcasts $\{y_i\}_{i=0}^n$ and $(d,z(X))$.
        \end{itemize}
        
        \vspace{0.5em}
        \textbf{Verification:}
            Given public keys $\{PK_i\}_{i=1}^n$ and public(commitment) key $PK_0$, any entity can check 
            $\pi_{PDL}$ to verify the correctness of the encrypted shares $\{y_i\}_{i=1}^n$ and $y_0$ being the encryption(/commitment) of the secret. 
            They will output \textbf{true} or \textbf{false} based on the verification of the proof. The 
            procedure is outlined as follows:
        \begin{itemize}
            \item The entity checks if $z(X)$ is a $t-$degree polynomial or not.
            \item Checks if $d=\mathcal{H}(y_0,y_1,\dots,y_n,\frac{PK_0^{z(0)}}{y_0^d},\frac{PK_1^{z(1)}}{y_1^d},\dots,\frac{PK_n^{z(n)}}{y_n^d})$.
            \item Outputs \textbf{true} if both of the above checks are satisfied, otherwise \textbf{false}.
        \end{itemize}

        \vspace{0.5em}
        \textbf{Reconstruction:}
            There are two approaches to reconstruct the secret $g^{f_0}$ based on the 
            cooperation of the dealer $D$ which are as follows:
            \begin{itemize}
                \item \textbf{Optimistic Reconstruction:} $D$ publishes $f_0$, then any verifier (not necessarily shareholders) 
                when given $g,PK_0,y_0$ can check if $y_0=PK_0^{f_0}$ and returns $g^{f_0}$ 
                if the check passes, if not they return \textbf{false}.
                \item \textbf{Pessimistic Reconstruction:} If $D$ refuses to reveal $f_0$, then any set 
                $\mathcal{Q}$ consisting at least $t+1$ shareholders will do the following (which 
                is same as the reconstruction step in the PVSS $\Pi_S$ protocol):
                \begin{itemize}
                    \item Each party $P_i\in\mathcal{Q}$ decrypts their share $y_i$ using their private key $SK_i$ 
                      corresponding to $PK_i$ to obtain $g^{f_i}$. Then they publish $g^{f_i}$ 
                      along with a DLEQ proof \ref{subsec:chaum-pedersen}, $\pi_{DLEQ}$ which proves that the 
                       $g^{f_i}$ is the correct decryption of $y_i$.
                    \item If $\mathcal{Q}$ consists at least $t+1$ honest parties, they can use the 
                    lagrange interpolation to compute the secret $g^{f_0}$ as follows:
                    \begin{align*}
                        g^{f_0} &= \prod_{i=1}^{t+1} g^{f_i^{\lambda_i}} = g^{\sum_{i=1}^{t+1}f_i\cdot\lambda_i},\\
                    \end{align*}
                    where $\lambda_i=\prod_{j\neq i}\frac{j}{j-i}$ are lagrange coefficients.
                \end{itemize}
            \end{itemize}
    \end{tcolorbox}
    }
    \caption[PPVSS Scheme]{a Pre-Constructed Publicly Verifiable Secret Sharing (PPVSS) scheme}
    \label{fig:ppvss}
\end{figure}