\begin{figure}[ht]
    \centering
    \begin{tcolorbox}[title=\textbf{Chaum-Pedersen Protocol for DLEQ}, width=0.9\textwidth, colframe=blue!75!black, colback=blue!10, sharp corners]
        Let $(g,h,a,b)\in L_{DLEQ}$ be a statement with its corresponding witness being $x$ where $L_{DLEQ}$
        is the language defined by the relation $R_{DLEQ}$ and $\mathcal{H}$ be a Random Oracle ($RO$).
        
        \vspace{0.5em}
        \textbf{Prover}
        \begin{itemize}
            \item Samples $r\in_{R}\mathbb{Z}_q$ uniformly at random and sets 
                $c_1=g^r$ and $c_2=h^r$.
            \item Sets $d\leftarrow \mathcal{H}(a,b,c_1,c_2)$, where $\mathcal{H}$ is 
                an agreed upon Random Oracle ($RO$).
            \item Sets $z\equiv r+dx \pmod{q}$ and returns the proof(/transcript) $\pi:= (d,z)$.
        \end{itemize}
        
        \vspace{0.5em}
        \textbf{Verifier}
        \begin{itemize}
            \item Checks if $d\leftarrow \mathcal{H}(a,b,\frac{g^z}{a^d},\frac{h^z}{b^d})$ 
                and outputs \textbf{true} or \textbf{false} accordingly.
        \end{itemize}
    \end{tcolorbox}
    \caption{Chaum-Pedersen NIZK PoK for DLEQ}
    \label{fig:chaum-pedersen}
\end{figure}
