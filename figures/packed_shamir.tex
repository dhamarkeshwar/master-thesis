\begin{figure}[ht]
    \centering
    \begin{tcolorbox}[title=\textbf{Packed Shamir Secret Sharing}, width=0.9\textwidth, colframe=blue!75!black, colback=blue!10, sharp corners]
        Given $\ell$ secrets to share amongst $n$ parties, where at most $t$ of them can
        be \textit{(passively)} corrupt, the $(n,t,\ell)$-Packed Shamir secret sharing scheme description is as
        follows:
        
        \vspace{0.5em}
        \textbf{Sharing Algorithm:}
        \begin{itemize}
            \item Dealer constructs the secret polynomial $f\in\mathbb{Z}_q[X]_{t+l-1}$
                  via the lagrange interpolation by choosing $t+\ell$ elements in 
                  $\mathbb{Z}_q$ where $\ell$ of them are secrets, $\{s_i\}_{i=0}^{\ell-1}$, with
                  $f(-i)=s_i$ for all $i$ and remaining $t$ are chosen uniformly 
                  at random in $\mathbb{Z}_q$.
            \item Each party $P_i$ receives their share $f(i)$ from the Dealer
                  for each $i\in\{1,\dots,n\}$
        \end{itemize}
        
        \vspace{0.5em}
        \textbf{Reconstruction Algorithm:}
        \begin{itemize}
            \item Any $\mathcal{Q}$ set containing at least $t+\ell$ parties can use the 
            lagrange interpolation to compute $\{s_i\}_{i=0}^{\ell-1}$ as follows:
            \begin{align*}
                s_m &= \sum_{i\in \mathcal{Q}} f(i) \left[\prod_{j\in \mathcal{Q}, j\neq i}\frac{-m-j}{i-j}\right] &&, m\in\{0,\dots,\ell-1\} \\
            \end{align*}
            \item The secrets $\{s_i\}_{i=0}^{\ell-1}$ are outputted as the result.
        \end{itemize}
    \end{tcolorbox}
    \caption{Packed Shamir Secret Sharing}
    \label{fig:packed-shamir}
\end{figure}