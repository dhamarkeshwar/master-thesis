\begin{figure}[ht]
    \centering
    \begin{tcolorbox}[title=$\pi_{PDL}^{mod-v1}$, width=0.9\textwidth, colframe=blue!75!black, colback=blue!10, sharp corners]
        Let $(\{g_i,x_i\}_{i=-(\ell-1)}^n,\{F_i\}_{i=0}^n)\in L_{PDL}^{mod-v1}$ be a statement with its corresponding witness being $f\in\mathbb{Z}_q[X]_{\leq t+\ell-1}$ 
        where $L_{PDL}^{mod-v1}$ is the language defined by the relation $R_{PDL}^{mod-v1}$ and $\mathcal{H}$ be a 
        random oracle (RO).
        
        \vspace{0.5em}
        \textbf{Prover}
        \begin{itemize}
            \item Samples $r\in_{R}\mathbb{Z}_q[X]_{t+\ell-1}$ uniformly at random and sets 
              $\Gamma=\prod_{i=-(\ell-1)}^{n}g_i^{r(x_i)}$.
            \item Sets $d\leftarrow \mathcal{H}(F_0,\Gamma)$, where $\mathcal{H}$ is 
                an agreed upon Random Oracle (RO).
            \item Sets $z(X)\equiv r(X)+df(X) \pmod{q}$ and returns the proof(/transcript) $\pi:= (d,z(X))$.
        \end{itemize}
        
        \vspace{0.5em}
        \textbf{Verifier}
        \begin{itemize}
            \item First checks if $z$ is at most a $t+\ell-1$ degree polynomial in $\mathbb{Z}_q[X]$. If so, they proceed with the next step.
            \item Checks if $d\leftarrow \mathcal{H}(F_0,\frac{\prod_{i=-(\ell-1)}^{n}g_i^{z(x_i)}}{F_0^d})$. 
            \item If first two steps are correct then they output \textbf{true}, otherwise \textbf{false}.
        \end{itemize}
    \end{tcolorbox}
    \caption{A NIZK AoK for Polynomial DL based on \ref{fig:polynomial-dl}}
    \label{fig:mod-v1-polynomial-dl}
\end{figure}
