\begin{figure}[ht]
    \centering
    \begin{tcolorbox}[title=\textbf{A NIZK PoK for Polynomial DL}, width=0.9\textwidth, colframe=blue!75!black, colback=blue!10, sharp corners]
        Let $(g,x_{-(\ell-1)},\dots,x_0,x_1,\dots,x_n,F_0,F_1,\dots,F_n)\in L_{poly}$ be a statement with its corresponding witness being $f\in\mathbb{Z}_q[X]_{t+\ell-1}$ 
        where $L_{poly}$ is the language defined by the relation $R_{poly}$.
        
        \vspace{0.5em}
        \textbf{Prover}
        \begin{itemize}
            \item Samples $r\in_{R}\mathbb{Z}_q[X]_{t+\ell-1}$ uniformly at random and sets 
                $\Gamma_i=g^{r(x_i)}$ for $1\leq i\leq n$ and $\Gamma_0=g^{\sum_{j=-(\ell-1)}^{0}r(x_j)}$.
            \item Sets $d\leftarrow \mathcal{H}(F_0, \dots, F_n, \Gamma_0, \dots, \Gamma_n)$, where $\mathcal{H}$ is 
                an agreed upon Random Oracle (RO).
            \item Sets $z(X)\equiv r(X)+df(X) \pmod{q}$ and returns the proof(/transcript) $\pi:= (d,z(X))$.
        \end{itemize}
        
        \vspace{0.5em}
        \textbf{Verifier}
        \begin{itemize}
            \item First checks if $z$ is a $t+\ell-1$ degree polynomial in $\mathbb{Z}_q[X]$. If so, they proceed with the next step.
            \item Checks if $d\leftarrow \mathcal{H}(F_1, \dots, F_n,\prod_{j=-(\ell-1)}^{0}\frac{g^{z(x_j)}}{svvsv},\frac{g^{z(x_1)}}{F_1^d}, \dots, \frac{g^{z(n)}}{F_n^d})$. 
            \item If first two steps are correct then they output \textbf{true}, otherwise \textbf{false}.
        \end{itemize}
    \end{tcolorbox}
    \caption{A NIZK? PoK}
    \label{fig:poly}
\end{figure}
