\documentclass[master=elt,masteroption=ec]{kulemt}
\setup{
  title={Beste masterproef ooit al geschreven},
  author={Een Auteur\and Tweede Auteur},
  promotor={Prof.\,dr.\,ir.\ Weet Beter},
  assessor={Ir.\,W. Eetveel\and W. Eetrest},
  assistant={Ir.\ A.~Assistent \and D.~Vriend}}
% Verwijder de "%" op de volgende lijn als je de kaft wil afdrukken
%\setup{coverpageonly}
% Verwijder de "%" op de volgende lijn als je enkel de eerste pagina's wil
% afdrukken en de rest bv. via Word aanmaken.
%\setup{frontpagesonly}

% Hier kun je dan nog andere pakketten laden of eigen definities voorzien

% Tenslotte wordt hyperref gebruikt voor pdf bestanden.
% Dit mag verwijderd worden voor de af te drukken versie.
\usepackage[pdfusetitle,colorlinks,plainpages=false]{hyperref}

%%%%%%%
% Om wat tekst te genereren wordt hier het lipsum pakket gebruikt.
% Bij een echte masterproef heb je dit natuurlijk nooit nodig!
\IfFileExists{lipsum.sty}%
 {\usepackage{lipsum}\SetLipsumDefault{11-13}}%
 {\newcommand{\lipsum}[1][11-13]{\par Hier komt wat tekst: lipsum ##1.\par}}
%%%%%%%

%\includeonly{hoofdstuk-n}
\begin{document}

\begin{preface}
  Dit is mijn dankwoord om iedereen te danken die mij bezig gehouden heeft.
  Hierbij dank ik mijn promotor, mijn begeleider en de voltallige jury.
  Ook mijn familie heeft mij erg gesteund natuurlijk.
\end{preface}

\tableofcontents*

\begin{abstract}
  In dit \texttt{abstract} environment wordt een al dan niet uitgebreide
  samenvatting van het werk gegeven. De bedoeling is wel dat dit tot
  1~bladzijde beperkt blijft.

  \lipsum[1]
\end{abstract}

% Een lijst van figuren en tabellen is optioneel
%\listoffigures
%\listoftables
% Bij een beperkt aantal figuren en tabellen gebruik je liever het volgende:
\listoffiguresandtables
% De lijst van symbolen is eveneens optioneel.
% Deze lijst moet wel manueel aangemaakt worden, bv. als volgt:
\chapter{Lijst van afkortingen en symbolen}
\section*{Afkortingen}
\begin{flushleft}
  \renewcommand{\arraystretch}{1.1}
  \begin{tabularx}{\textwidth}{@{}p{12mm}X@{}}
    LoG   & Laplacian-of-Gaussian \\
    MSE   & Mean Square error \\
    PSNR  & Peak Signal-to-Noise ratio \\
  \end{tabularx}
\end{flushleft}
\section*{Symbolen}
\begin{flushleft}
  \renewcommand{\arraystretch}{1.1}
  \begin{tabularx}{\textwidth}{@{}p{12mm}X@{}}
    42    & ``The Answer to the Ultimate Question of Life, the Universe,
            and Everything'' volgens de \cite{h2g2} \\
    $c$   & Lichtsnelheid \\
    $E$   & Energie \\
    $m$   & Massa \\
    $\pi$ & Het getal pi \\
  \end{tabularx}
\end{flushleft}

% Nu begint de eigenlijke tekst
\mainmatter

\chapter{Inleiding}
\label{inleiding}
In dit hoofdstuk wordt het werk ingeleid. Het doel wordt gedefinieerd en er
wordt uitgelegd wat de te volgen weg is (beter bekend als de rode draad).

Als je niet goed weet wat een masterproef is, kan je altijd
Wikipedia\cite{wiki} eens nakijken.

\section{Lorem ipsum 4--5}
\lipsum[4-5]

\section{Lorem ipsum 6--7}
\lipsum[6-7]

\chapter{Het eerste hoofdstuk}
\label{hoofdstuk:1}
Een hoofdstuk behandelt een samenhangend geheel dat min of meer op zichzelf
staat. Het is dan ook logisch dat het begint met een inleiding, namelijk
het gedeelte van de tekst dat je nu aan het lezen bent.

\section{Eerste onderwerp in dit hoofdstuk}
De inleidende informatie van dit onderwerp.

\subsection{Een item}
De bijbehorende tekst. Denk eraan om de paragrafen lang genoeg te maken en
de zinnen niet te lang.

Een paragraaf omvat een gedachtengang en bevat dus steeds een paar zinnen.
Een paragraaf die maar één lijn lang is, is dus uit den boze.

\section{Tweede onderwerp in dit hoofdstuk}
Er zijn in een hoofdstuk verschillende onderwerpen. We zullen nu
veronderstellen dat dit het laatste onderwerp is.

\subsection{Een item}
Maak ook geen misbruik van opsommingen. Voor korte opsommingen gebruik je
geen ``\verb|itemize|'' of ``\texttt{enumerate}'' commando's. Doe dus
\emph{niet} het volgende:
\begin{quote}
  De Eiffeltoren bevat drie verdiepingen:
  \begin{itemize}
  \item de eerste;
  \item de tweede;
  \item de derde.
  \end{itemize}
\end{quote}
Maar doe:
\begin{quote}
  De Eiffeltoren bevat drie verdiepingen: de eerste, de tweede en de derde.
\end{quote}

\section{Besluit van dit hoofdstuk}
Als je in dit hoofdstuk tot belangrijke resultaten of besluiten gekomen
bent, dan is het ook logisch om het hoofdstuk af te ronden met een
overzicht ervan. Voor hoofdstukken zoals de inleiding en het
literatuuroverzicht is dit niet strikt nodig.

\chapter{Een volgend hoofdstuk}
\label{hoofdstuk:2}
Een hoofdstuk behandelt een samenhangend geheel dat min of meer op zichzelf
staat. Het is dan ook logisch dat het begint met een inleiding, namelijk
het gedeelte van de tekst dat je nu aan het lezen bent.

\section{Eerste onderwerp in dit hoofdstuk}
De inleidende informatie van dit onderwerp.

\subsection{Een item}
Een tekst staat nooit alleen. Dit wil zeggen dat er zeker ook referenties
nodig zijn. Dit kan zowel naar on-line documenten\cite{wiki} als naar
boeken\cite{pratchett06:_good_omens}.

\section{Figuren}
Figuren worden gebruikt om illustraties toe te voegen. Dit is dan ook de
manier om beeldmateriaal toe te voegen zoals getoond wordt in
figuur~\ref{fig:logo}.

\begin{figure}
  \centering
  \includegraphics{logokul}
  \caption{Het KU~Leuven logo.}
  \label{fig:logo}
\end{figure}

\section{Tabellen}
Tabellen kunnen gebruikt worden om informatie op een overzichtelijke te
groeperen. Een tabel is echter geen rekenblad! Vergelijk maar eens
tabel~\ref{tab:verkeerd} en tabel~\ref{tab:juist}. Welke tabel vind jij het
duidelijkst?

\begin{table}
  \centering
  \begin{tabular}{||l|lr||} \hline
    gnats     & gram      & \$13.65 \\ \cline{2-3}
              & each      & .01 \\ \hline
    gnu       & stuffed   & 92.50 \\ \cline{1-1} \cline{3-3}
    emu       &           & 33.33 \\ \hline
    armadillo & frozen    & 8.99 \\ \hline
  \end{tabular}
  \caption{Een tabel zoals het niet moet.}
  \label{tab:verkeerd}
\end{table}

\begin{table}
  \centering
  \begin{tabular}{@{}llr@{}} \toprule
    \multicolumn{2}{c}{Item} \\ \cmidrule(r){1-2}
    Animal    & Description & Price (\$)\\ \midrule
    Gnat      & per gram    & 13.65 \\
              & each        & 0.01 \\
    Gnu       & stuffed     & 92.50 \\
    Emu       & stuffed     & 33.33 \\
    Armadillo & frozen      & 8.99 \\ \bottomrule
  \end{tabular}
  \caption{Een tabel zoals het beter is.}
  \label{tab:juist}
\end{table}

\section{Lorem ipsum}
Tenslotte gaan we hier nog wat tekst voorzien zodat er minstens een
bijkomende bladzijde aangemaakt wordt. Dat geeft de gelegenheid om eens te
zien hoe de koptekst en de voettekst zich gedragen.

\subsection{Lorem ipsum dolor sit amet, consectetur adipiscing elit}
Sed nec tortor id felis tristique sodales. Nulla nec massa eu dui fermentum
tincidunt. Integer ullamcorper ante eget eros posuere faucibus. Nam id
ligula ut augue pulvinar vulputate id at purus. Aenean condimentum tortor
eu mi placerat eget eleifend massa mollis. Nam est mi, sagittis quis
euismod eget, sagittis in nibh. Proin elit turpis, aliquam et imperdiet
sed, volutpat eu turpis.

Pellentesque vel enim tellus, vitae egestas turpis. Praesent malesuada elit
non nisi sollicitudin non blandit lacus tincidunt. Morbi blandit urna at
lectus ornare laoreet. Suspendisse turpis diam, lobortis dictum luctus
quis, commodo at lorem. Integer lacinia convallis ultricies. Sed quis augue
neque, eu malesuada arcu. Nullam vehicula, purus vitae sagittis pulvinar,
erat eros semper massa, eu egestas nibh erat quis magna. Cras pellentesque,
nisl eu dapibus volutpat, urna augue ornare quam, quis egestas lectus nulla
a lectus.

Vivamus dictum libero in massa cursus sed vulputate eros imperdiet. Donec
lacinia, libero ac lobortis egestas, nibh dui ornare arcu, luctus porttitor
velit massa sit amet quam. Maecenas scelerisque laoreet diam, vitae congue
quam adipiscing vitae. Aliquam cursus nisl a leo convallis eleifend
fermentum massa porta. Nunc libero quam, dapibus dapibus molestie sit amet,
faucibus vel nunc.

\subsection{Praesent auctor venenatis posuere}
Sed tellus augue, molestie in pulvinar lacinia, dapibus non ipsum. Fusce
vitae mi vitae enim ullamcorper hendrerit eu malesuada est. Proin iaculis
ante sed nibh tincidunt vel interdum libero posuere. Vivamus accumsan metus
quis felis congue suscipit dapibus enim mattis. Fusce mattis tortor eget
ipsum interdum sagittis auctor id metus.

Integer diam lacus, pharetra sit amet tempor et, tristique non lorem.
Aenean auctor, nisi eu interdum fermentum, lectus massa adipiscing elit,
sed facilisis orci odio a lectus. Proin mi nibh, tempus quis porta a,
viverra quis enim. In sollicitudin egestas libero, quis viverra velit
molestie eget. Nulla rhoncus, dolor a mollis vestibulum, lacus elit semper
nisi, nec sollicitudin sem urna eu magna. Nunc sed est urna, euismod congue
mi.

\subsection{Cras vulputate ultricies venenatis}
Vivamus eros urna, sodales accumsan semper vel, lobortis sit amet mauris.
Etiam condimentum eleifend lorem, ullamcorper ornare lectus aliquet vitae.
Praesent massa enim, interdum sit amet semper et, venenatis ut elit.
Quisque faucibus, quam ac lacinia imperdiet, nulla neque elementum purus,
tempus rutrum justo massa porta sapien. Vestibulum ante ipsum primis in
faucibus orci luctus et ultrices posuere cubilia Curae; Sed ultrices
interdum mi, et rhoncus sapien rutrum sed.

Duis elit orci, molestie quis sollicitudin sed, convallis non ante.
Maecenas tincidunt condimentum justo, et ultricies leo tristique vitae.
Vestibulum quis quam non lectus dapibus eleifend a vitae nibh. Nam nibh
justo, pharetra quis iaculis consequat, elementum quis justo. Etiam mollis
lacinia lacus, nec sollicitudin urna lobortis ac. Nulla facilisi.

Proin placerat risus eleifend erat ultricies placerat. Etiam rutrum magna
nec turpis euismod consectetur. Phasellus tortor odio, lacinia imperdiet
condimentum sed, faucibus commodo erat. Phasellus sed felis id ante
placerat ultrices. Aenean tempor justo in tortor volutpat eu auctor dolor
mollis. Aenean sit amet risus urna. Morbi viverra vehicula cursus.

\subsection{Donec nibh ante, consectetur et posuere id, tempus nec arcu}
Curabitur a tellus aliquet ipsum pellentesque scelerisque. Etiam congue,
risus et volutpat rutrum, est purus dapibus leo, non cursus metus felis
eget ligula. Vivamus facilisis tristique turpis, ut pretium lectus luctus
eleifend. Fusce magna sapien, ullamcorper vitae fringilla id, euismod quis
ante.

Phasellus volutpat, nunc et pharetra semper, sem justo adipiscing mauris,
id blandit magna quam et orci. Vestibulum a erat purus, ut molestie ante.
Vestibulum ante ipsum primis in faucibus orci luctus et ultrices posuere
cubilia Curae; Proin turpis diam, consequat ut ullamcorper ut, consequat eu
orci. Sed metus risus, fringilla nec interdum vel, interdum eu nunc.
Suspendisse vel sapien orci.

\subsection{Morbi et mauris tempus purus ornare vehicula}
Mauris sit amet diam quam, eget luctus purus. Sed faucibus, risus semper
eleifend iaculis, mi turpis bibendum nisl, quis cursus nibh nisl sit amet
ipsum. Vestibulum tempor urna vitae mi auctor malesuada eget non ligula.
Nullam convallis, diam vel ultrices auctor, eros eros egestas elit, sed
accumsan arcu tortor eget leo. Vestibulum orci purus, porttitor in pharetra
eget, tincidunt eget nisl. Nullam sit amet nulla dui, facilisis vestibulum
dui.

Donec faucibus facilisis mauris ac cursus. Duis rhoncus quam sed nisi
laoreet eu scelerisque massa tincidunt. Vivamus sit amet libero nec arcu
imperdiet tempor quis non libero. Sed consequat dignissim justo. Phasellus
ullamcorper, velit quis posuere vulputate, felis erat tincidunt mauris, at
vestibulum justo lectus et turpis. Maecenas lacinia convallis euismod.
Quisque egestas fermentum sapien eu dictum. Sed nec lacus in purus dictum
consequat quis vel nisl. Fusce non urna sem. Curabitur eu diam vitae elit
accumsan blandit. Nullam fermentum nunc et leo dictum laoreet. Donec semper
varius velit vel fringilla. Vivamus eu orci nunc.

\section{Besluit van dit hoofdstuk}
Als je in dit hoofdstuk tot belangrijke resultaten of besluiten gekomen
bent, dan is het ook logisch om het hoofdstuk af te ronden met een
overzicht ervan. Voor hoofdstukken zoals de inleiding en het
literatuuroverzicht is dit niet strikt nodig.

% ... en zo verder tot
\chapter{Het laatste hoofdstuk}
\label{hoofdstuk:n}
Een hoofdstuk behandelt een samenhangend geheel dat min of meer op zichzelf
staat. Het is dan ook logisch dat het begint met een inleiding, namelijk
het gedeelte van de tekst dat je nu aan het lezen bent.

\section{Eerste onderwerp in dit hoofdstuk}
De inleidende informatie van dit onderwerp.

\subsection{Een item}
De bijbehorende tekst. Denk eraan om de paragrafen lang genoeg te maken en
de zinnen niet te lang.

Een paragraaf omvat een gedachtengang en bevat dus steeds een paar zinnen.
Een paragraaf die maar één lijn lang is, is dus uit den boze.

\section{Tweede onderwerp in dit hoofdstuk}
Er zijn in een hoofdstuk verschillende onderwerpen. We zullen nu
veronderstellen dat dit het laatste onderwerp is.

\section{Besluit van dit hoofdstuk}
Als je in dit hoofdstuk tot belangrijke resultaten of besluiten gekomen
bent, dan is het ook logisch om het hoofdstuk af te ronden met een
overzicht ervan. Voor hoofdstukken zoals de inleiding en het
literatuuroverzicht is dit niet strikt nodig.

\chapter{Besluit}
\label{besluit}
De masterproeftekst wordt afgesloten met een hoofdstuk waarin alle
besluiten nog eens samengevat worden. Dit is ook de plaats voor suggesties
naar het verder gebruik van de resultaten, zowel industriële toepassingen
als verder onderzoek.

\lipsum[1-7]


% Indien er bijlagen zijn:
\appendixpage*          % indien gewenst
\appendix
\chapter{De eerste bijlage}
\label{app:A}
In de bijlagen vindt men de data terug die nuttig kunnen zijn voor de
lezer, maar die niet essentieel zijn om het betoog in de normale tekst te
kunnen volgen. Voorbeelden hiervan zijn bronbestanden,
configuratie-informatie, langdradige wiskundige afleidingen, enz.

In een bijlage kunnen natuurlijk ook verdere onderverdelingen voorkomen,
evenals figuren en referenties\cite{h2g2}.

\section{Meer lorem}
\lipsum[50]

\subsection{Lorem 15--17}
\lipsum[15-17]

\subsection{Lorem 18--19}
\lipsum[18-19]

\section{Lorem 51}
\lipsum[51]

% ... en zo verder tot
\chapter{De laatste bijlage}
\label{app:n}
In de bijlagen vindt men de data terug die nuttig kunnen zijn voor de
lezer, maar die niet essentieel zijn om het betoog in de normale tekst te
kunnen volgen. Voorbeelden hiervan zijn bronbestanden,
configuratie-informatie, langdradige wiskundige afleidingen, enz.

\section{Lorem 20-24}
\lipsum[20-24]

\section{Lorem 25-27}
\lipsum[25-27]


\backmatter
% Na de bijlagen plaatst men nog de bibliografie.
% Je kan de standaard "abbrv" bibliografiestijl vervangen door een andere.
\bibliographystyle{abbrv}
\bibliography{referenties}

\end{document}
