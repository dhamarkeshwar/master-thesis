\newcommand*\showtemplatefile[1]{%
  \begingroup
    \setverbatimfont{\normalfont\ttfamily\footnotesize}%
    \bvperpagefalse
    \linenumberfrequency{1}\resetbvlinenumber
    \linenumberfont{\rmfamily\tiny}%
    \bvnumbersoutside
    \renewcommand*\bvtoprulehook{%
      \hrule width\linewidth \nobreak
      \hbox to\linewidth{%
        \hss \smash{\colorbox[gray]{.9}{%
            \ttfamily \enspace template/#1\enspace}}\qquad}%
      \nobreak\vskip-.1pt\nobreak\vskip-\lineskip\relax}%
    \boxedverbatiminput{#1}%
    \tracingnone
  \endgroup}

\chapter{LaTeX Template}
\label{app:template}
LaTeX templates are provided for a Dutch master's programme as well as for an
English one. The Dutch and English examples can be found respectively in the
directories `\file{sjabloon}' and `\file{template}' of the documentation
directory (the directory where the file you're reading is installed). For each
main file (either \file{masterproef.tex} or \file{thesis.tex}) the resulting
typeset \PDF\ file is available as a reference.

The following sections give examples of the most important files of the
template: the main file (\Sref{sec:file:main}), a sample chapter
(\Sref{sec:file:chap}), and the bibliography database (\Sref{sec:file:bib}).
Translated versions for a Dutch master's programme can be found in the directory
`\file{sjabloon}'. Finally \Sref{sec:file:hoofd} gives an example of a main
file for a Dutch master's programme but written in English.

All the examples below use the \pkg{lipsum} package to generate dummy text.
Of course this is only needed as an example to quickly generate a lot of
text. Because you will never need it in your real text, you can start by
removing all invocations of \cs{lipsum} in all files as well as the lines
20--27 of the main files (\Sref{sec:file:main} and \Sref{sec:file:hoofd}).

\section{The main file}
\label{sec:file:main}
\showtemplatefile{thesis.tex}

\section{A sample chapter}
\label{sec:file:chap}
\showtemplatefile{chapter-1.tex}

\section{The bibliography database}
\label{sec:file:bib}
\showtemplatefile{references.bib}

\section{The main file of an English text for a Dutch master's programme}
\label{sec:file:hoofd}

The main differences between an English text for an English master's programme
(\Sref{sec:file:main}) and an English text for an Dutch master's programme are:
\begin{itemize}
\item the use of the \opt{english} option (line~1);
\item an additional \env{abstract*} environment is needed (lines 51--59).
\end{itemize}
\medskip

\showtemplatefile{thesis-dutch.tex}

%%% Local Variables: 
%%% mode: latex
%%% TeX-master: "kulemt"
%%% ispell-local-dictionary: "british"
%%% End: 
